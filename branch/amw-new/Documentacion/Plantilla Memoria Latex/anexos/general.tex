% ------------------------------------------------------------------------------
% Este fichero es parte de la plantilla LaTeX para la realización de Proyectos
% Final de Grado, protegido bajo los términos de la licencia GFDL.
% Para más información, la licencia completa viene incluida en el
% fichero fdl-1.3.tex

% Copyright (C) 2012 SPI-FM. Universidad de Cádiz
% ------------------------------------------------------------------------------


\chapter*{\rlap{Información sobre Licencia}}
\phantomsection  % so hyperref creates bookmarks
\addcontentsline{toc}{chapter}{Información sobre Licencia}
%\label{label_fdl}

 \begin{center}

      \textbf{Información sobre Licencia}


\end{center}

El producto está licenciado bajo GPL 1.3 y a continuación se expone su licencia.\\

\textbf{Licencia pública general GNU}\\

Versión 3, 29 de junio de 2007\\

Copyright (C) 2007 Free Software Foundation, Inc. \url{http://fsf.org/}\\

Se  permite  la  copia  y  distribución  de  copias  literales  de  esta  licencia,  pero  no está permitido modificarla\\

\textbf{Preámbulo}\\

La Licencia Pública General GNU es una licencia libre, de copia permitida (copy-left) para software y otros tipos de trabajos.\\

Las licencias de la mayoría del software y otros trabajos prácticos están diseñadas para quitarle a usted la libertad de compartir y modificar esos trabajos. Por el contrario, la Licencia Pública General GNU pretende garantizarle la libertad para compartir y modificar todas las versiones de un programa y asegurar que permanecerá como software libre para todos sus usuarios. Nosotros, La Fundación Software Libre, usamos la Licencia Pública General GNU para la mayoría de nuestro software; también se aplica a cualquier otro trabajo liberado de esta forma por sus autores. Usted también puede aplicarla a sus programas.\\

Cuando hablamos de software libre, nos referimos a libertad, no al precio. Nuestras Licencias Públicas Generales están diseñadas para asegurar que usted tenga la libertad  para  distribuir  copias  de  software  libre  (y  cobrar  por  ello  si  quiere),  que reciba  el  código  fuente  o  pueda  obtenerlo  si  usted  quiere,  que  pueda  modificar  el
software o usar parte del mismo en nuevos programas libres, y que sepa que pueda hacer estas cosas.\\

Para proteger sus derechos, necesitamos impedir que otros le nieguen estos derechos o le soliciten renunciar a ellos. Por lo tanto, usted tiene ciertas responsabilidades si distribuye copias del software, o si usted lo modifica: la responsabilidad para respetar la libertad de otros.\\

Por ejemplo, si distribuye copias de tales programas, gratuitamente o no, debe transmitir a los destinatarios los mismos derechos que usted recibió. Debe asegurarse que ellos también reciban o puedan conseguir el código fuente. Y debe mostrarles estas condiciones para que conozcan sus derechos.\\

Los desarrolladores que usen la GPL GNU le protegen sus derechos de dos formas: (1) imponen derechos al software, y (2) le ofrecen esta Licencia para que legalmente lo copie, distribuya y/o modifique.\\

Para proteger a desarrolladores y autores, la GPL expone claramente que no existe  garantía  alguna  para  este  software  libre.  Para  beneficio  de  ambos,  usuarios  y autores, la GPL establece que las versiones modificadas deben estar identificadas de forma apropiada como tales, para que cualquier problema no sea atribuido por error
a los autores de versiones anteriores.\\

Algunos dispositivos son diseñados para negar al usuario la instalación o ejecución de versiones modificadas del software que usan internamente, aunque el fabricante sí puede hacerlo. Esto es completamente incompatible con el objetivo de proteger la libertad de los usuarios para modificar el software. Este tipo de abuso sistemático
ocurre con productos de consumo para uso personal, que es precisamente en donde es menos aceptable. Por tanto, hemos diseñado esta versión de la GPL para prohibir estas prácticas en esos productos. Si apareciesen problemas similares en otros ámbitos, estaremos atentos para extender estas prestaciones en las próximas versiones de la GPL, tanto como sea necesario para proteger la libertad de los usuarios.\\

Por último, todo programa está constantemente amenazado por las patentes de software. Los estados no deberán permitir que las patentes restrinjan el desarrollo y  el  uso  de  software  en  ordenadores  de  propósito  general,  pero  en  aquellos  que  lo hagan, esperamos evitar el especial peligro que suponen las patentes, que aplicadas
a un programa libre puedan hacerlo propietario en la práctica. Para prevenir eso, la GPL establece que las patentes no pueden usarse para convertir un programa en no-libre.\\

A continuación siguen los términos precisos y las condiciones para la copia, distribución y modificación.\\

\textbf{Términos y condiciones para la copia, distribución y modificación}

\begin{itemize}
	
	\item En adelante "Esta Licencia" se refiere a la versión 3 de la Licencia Publica General GNU.
	
	\item "Copyright" también significa "leyes similares al copyright" que son aplicables a otro tipo de trabajos, como las mascaras de semiconductores.\\
	
	\item "El Programa" se refiere a cualquier trabajo con copyright al que se haya aplicado esta Licencia. Cada concesionario es asimilable a "usted". "Concesionarios" y	"destinatarios" pueden ser personas físicas u organizaciones.\\
	
	\item "Modificar" un trabajo significa copiar o adaptar en su totalidad o parcialmente	un trabajo de manera que requiera permiso de copyright, de cualquier tipo salvo la copia exacta. El trabajo resultante se denomina "versión modificada" de un trabajo anterior o trabajo "basado en" el trabajo anterior.\\
	
	\item Un "trabajo amparado" puede ser tanto el Programa no modificado como un trabajo basado en el Programa.\\
	
	\item "Difundir" un trabajo significa hacer cualquier cosa con el, sin permiso, que le haga directa o indirectamente responsable de infringir leyes cubiertas por copyright, excepto la ejecución en un ordenador o la modificación de una copia privada. La difusión incluye la copia, distribución (con o sin modificaciones), distribución pública,
	y en algunos países también otras actividades.\\
	
	\item "Transmitir" un trabajo significa cualquier tipo de difusión que permite a la otra parte hacer o recibir copias. La mera interacción con un usuario mediante una red de ordenadores, sin transferir copia alguna, no se considera "transmisión".\\

\end{itemize}

 Una interfaz de usuario interactivo muestra "Avisos Legales Apropiados" siempre	y cuando incluya características visuales apropiadas y destacadas que (1) muestren un aviso de copyright apropiado, y (2) indiquen al usuario que no existe garantía alguna para el trabajo (exceptuando las garantías que se hayan podido establecer), que  los  concesionarios  deben  acompañar  el  trabajo  con  esta  Licencia,  y  cómo  se puede ver una copia de esta Licencia. Si la interfaz muestra una lista de opciones o comandos, tales como menús, un elemento destacado en dicha lista cumple estos criterios.\\

\begin{enumerate}
	
	\item Código Fuente.\\
	
	El "código fuente" de un trabajo es el formato preferido para realizar modificaciones en él. "Código objeto" se refiere a cualquier formato del trabajo que no sea código fuente.\\
	
	Una "Interfaz Estándar" se refiere a una interfaz que sea o bien un estándar oficial definido por una institución de estándares reconocida, o bien, en el caso de interfaces específicas para un determinado lenguaje de programación, uno cuyo uso esté extendido entre los desarrolladores que trabajan con ese lenguaje.\\
	
	Las "Bibliotecas de Sistema" de un trabajo ejecutable incluyen cualquiera, que no sea el trabajo completo, que (a) esté incluida de la misma forma que un componente principal, pero que no forme parte de ese componente principal, y (b) sólo sirva para habilitar la utilización del trabajo a través de ese componente principal, o para implementar una Interfaz Estándar para el cual está disponible una implementación pública en código fuente. Un "Componente Principal", en este contexto, se refiere a un componente principal y esencial (núcleo, sistema de ventanas y similares) del sistema operativo particular (en su caso) sobre el cual funcione el ejecutable, o un compilador utilizado para generar el trabajo, o un intérprete de código objeto utilizado para ejecutarlo.\\
	
	La "Fuente Correspondiente" de un trabajo en código objeto se refiere a todo el código fuente necesario para generar, instalar, y (para un trabajo ejecutable) ejecutar el código objeto y modificar el trabajo, incluyendo guiones que controlen esas actividades. De todas formas, no se incluyen las Bibliotecas de Sistema del trabajo, o herramientas de propósito general o programas gratuitos habitualmente disponibles y usados para realizar estas actividades que no forman parte del trabajo. Por ejemplo, la Fuente Correspondiente incluye los archivos de definición de interfaz asociados con archivos fuente del trabajo, y el código fuente de las bibliotecas compartidas o subprogramas enlazados dinámicamente que el programa requiere por diseño, como la comunicación de datos intrínseca o el control de  flujo entre esos subprogramas y otras partes del trabajo.\\
	
	La Fuente Correspondiente no incluye necesariamente aquello que los usuarios pueden regenerar automáticamente desde otras partes de la Fuente Correspondiente.\\
	
	La Fuente Correspondiente de un trabajo en código fuente es ese mismo trabajo.\\
	
	\item Permisos Básicos.\\
	
	Todos los derechos garantizados por esta Licencia se otorgan como copyright del Programa, y se proporcionan de manera irrevocable siempre y cuando se cumplan las condiciones establecidas. ésta Licencia afirma explícitamente su permiso ilimitado para ejecutar el Programa sin modificaciones. El resultado de la ejecución de un programa amparado está cubierto por esta Licencia sólo si la salida, por su contenido, constituye un trabajo amparado. Esta Licencia reconoce sus derechos por uso razonable u otro equivalente, tal y como determina la ley de copyright.\\
	
	Usted puede realizar, ejecutar y difundir trabajos amparados que no transmite, sin condición alguna mientras no tenga otra licencia más restrictiva. Puede transportar trabajos amparados a terceros con el mero objetivo de que ellos hagan modificaciones exclusivamente para usted, o para que le proporcionen ayuda para ejecutar esos trabajos, siempre que cumpla los términos de esta Licencia transportando todo el material de cuyo copyright no posee el control. Aquellos que realicen o ejecuten los trabajos amparados para usted deben hacerlo exclusivamente en su nombre, bajo su dirección y control, con términos que les prohíban realizar copias de su material con copyright al margen de su relación con usted.\\
	
	El transmisión bajo otras circunstancias se permite únicamente bajo las condiciones establecidas más abajo. No está permitido sublicenciar; la sección 10 lo hace innecesario.\\
	
	\item Protección de Derechos Legales de Usuarios frente a Leyes Anti-Evasión.\\
	
	Ningún trabajo amparado debe considerarse parte de una medida tecnológica efectiva, bajo cualquier ley aplicable que cumpla las obligaciones del artículo 11 del tratado de copyright WIPO adoptado el 20 de diciembre de 1996, o leyes similares que prohíben o restringen la evasión de tales medidas.\\
	
	Cuando transmite un trabajo amparado, usted renuncia a cualquier poder legal para prohibir la evasión de medidas tecnológicas mientras tales evasiones se realicen en ejercicio de derechos amparados por esta Licencia respecto al trabajo amparado, y usted niega cualquier intención de limitar el uso o modificación del trabajo con el objetivo de imponer, al trabajo de los usuarios, sus derechos legales o de terceros para prohibir la evasión de medidas tecnológicas.\\
	
	\item Transmisión de copias literales.\\
	
	Usted puede transmitir copias literales del código fuente del Programa tal y como lo recibe, por cualquier medio, siempre que publique de forma clara y llamativa en cada copia un aviso de copyright apropiado; mantenga intactos todos los avisos que afirmen que esta Licencia y cualquier término no-permisivo añadido y acorde con la sección 7 son aplicables al código; mantenga intactos todos los avisos de ausencia de garantía; y proporcione a todos los destinatarios una copia de esta Licencia junto con el Programa.\\
	
	Usted puede cobrar cualquier importe o no cobrar nada por cada copia que transmite, y puede ofrecer soporte o protección de garantía mediante un pago.\\
	
	\item Transmisión de Versiones Modificadas de Código.\\
	
	Usted puede transmitir un trabajo basado en el Programa, o las modificaciones que lo producen a partir del Programa, como código fuente bajo los términos de la sección 4, siempre que cumpla todas las condiciones siguientes:\\
	
	\begin{enumerate}
		
		\item El trabajo debe incluir avisos destacados indicando que usted lo ha modificado, dando una fecha pertinente.\\
		
		\item El trabajo debe incluir avisos destacados indicando que está realizado bajo esta Licencia y cualquier otra condición añadida bajo la sección 7.\\
		
		\item Usted debe aplicar la licencia al trabajo completo, como un todo, a cualquier persona que tenga posesión de una copia. Esta Licencia se aplicará por tanto, al igual que cualquier otra condición adicional de la sección 7, al conjunto completo del trabajo y todas y cada una de sus partes, independientemente de como sean agrupadas o empaquetadas. Esta Licencia no permite ser aplicada al trabajo de ninguna otra forma, pero no se anula dicho permiso si usted lo ha recibido por separado.\\
		
		\item Si el trabajo tiene interfaces de usuario interactivas, cada uno debe mostrar Avisos Legales Apropiados; de todas formas, si el Programa tiene interfaces interactivas que no muestran Avisos Legales Apropiados, su trabajo no tiene porqué modificarlos para que lo hagan.\\
		
	\end{enumerate}
	
	Un conjunto o recopilación formado por un trabajo amparado y otros trabajos distintos e independientes, que por su naturaleza no sean extensiones del trabajo amparado, y que no se combinen con él de alguna forma para dar lugar a un programa mayor, ubicados en un medio de distribución o almacenamiento, es llamado un "paquete" si la recopilación y su copyright al completo no son usados para limitar los derechos legales de los usuarios de la recopilación, más allá de lo que el trabajo individual permita. La inclusión de un trabajo amparado en un paquete no hace aplicable esta Licencia al resto de elementos del paquete.\\
	
	\item Transmisión de código No-fuente.\\
	
	Usted puede transmitir el código objeto de un trabajo amparado bajo los términos de las secciones 4 y 5 siempre que también transmite las Fuentes Correspondientes en lenguaje máquina bajo los términos de esta Licencia de alguna de las siguientes maneras:\\
	
	\begin{enumerate}
		
		\item Transmitir el código objeto en, o incorporado en, un producto físico (incluyendo medios de distribución físicos), acompañado por las Fuentes Correspondientes en un medio físico duradero habitual para el intercambio de software.\\
		
		\item Transmitir el código objeto en, o incorporado en, un producto físico (incluyendo medios de distribución físicos), acompañado de una oferta por escrito, válida al menos durante tres años y válida durante el tiempo que usted ofrezca recambios o soporte para ese modelo de producto, que ofrezca al poseedor del código objeto (1) una copia de las Fuentes Correspondientes a todo el software del producto que esté cubierto por esta Licencia, en un medio físico duradero habitual para el intercambio de software, a un precio no mayor que su coste razonable por transmitir físicamente las fuentes, o (2) acceder a copiar las fuentes correspondientes desde un servidor de red sin coste alguno.\\
		
		\item Transmitir copias individuales del código objeto junto a una copia de una oferta por escrito para proporcionar las Fuentes Correspondientes. Esta alternativa sólo está permitida ocasional y no comercialmente, y solamente si usted recibió el código objeto junto a una oferta parecida, de acuerdo con la subsección 6b.\\
		
		\item Transmitir el código objeto ofreciendo acceso desde un lugar determinado (gratuitamente o mediante pago), y ofrecer acceso equivalente a las Fuentes Correspondientes de la misma forma y el mismo lugar sin cargo añadido. No es necesario requerir a los destinatarios copiar las Fuentes Correspondientes junto al código objeto. Si el lugar para copiar el código objeto es un servidor de red, las Fuentes Correspondientes pueden estar en un servidor diferente (gestionado por usted o tercero) que ofrezca facilidades de copia equivalentes, siempre que mantenga instrucciones claras junto al código objeto sobre dónde encontrar las Fuentes Correspondientes. Independientemente de qué servidores alberguen las Fuentes Correspondientes, usted sigue obligado a asegurar que estarán disponibles durante el tiempo necesario para cumplir estos requisitos.\\
		
		\item Transmitir el código mediante transferencias entre usuarios, siempre que informe a otros usuarios dónde se ofrecen el código objeto y las Fuentes Correspondientes de forma pública sin cargo alguno bajo tal y como establecer la subsección 6d. Una parte separable del código objeto, cuyo código fuente esté excluido de las Fuentes Correspondientes como Biblioteca de Sistema, no necesita ser incluida en el transmisión del código objeto del trabajo.\\
		
	\end{enumerate}
	
	Un "Producto de Usuario" es tanto (1) un "producto de consumo", que se refiere a cualquier propiedad personal tangible habitualmente utilizada para fines personales, familiares o domésticos, o (2) cualquier cosa diseñada o vendida para ser incorporada como extensión. Para determinar si un producto es un producto de consumo, los casos dudosos se resolverán favoreciendo el amparo. Para un producto en particular recibido por un usuario particular, "de uso habitual" se refiere al uso típico o corriente de ese tipo de producto, independientemente de la situación del usuario particular o de la forma en que el usuario particular utilice, o pretenda o se espere que pretenda utilizar, el producto. Un producto es un producto de consumo independientemente de si el producto tiene usos sustancialmente comerciales, industriales o no de consumo, a menos que tales usos representen el único tipo significativo de uso del producto.\\
	
	La "Información de Instalación" para un Producto de Usuario se refiere a cualquier método, procedimiento, clave de autorización, u otro tipo de información necesaria para instalar y ejecutar una versión modificada de un trabajo amparado en ese Producto de Usuario a partir de una versión modificada de las Fuentes Correspondientes. La información debe ser suficiente para asegurar el funcionamiento continuo del código objeto modificado sin ningún tipo de condicionamiento o intromisión por el simple hecho de haber sido modificado.\\
	
	Si usted transmite bajo las premisas de esta sección el código objeto de un trabajo en, o con, o específicamente para ser usado en un Producto de Usuario, y el transmisión forma parte de una transacción donde los derechos de posesión y uso del Producto de Usuario se transfieren al destinatario a perpetuidad o durante un plazo fijo de tiempo (independientemente de las características de la transacción), las Fuentes Correspondientes transmitidas bajo estos supuestos debe ser acompañada de la Información de Instalación. Estos requerimientos no se aplican si ni usted ni terceros tienen posibilidad de instalar código objeto modificado en el Producto de Usuario (por ejemplo, el trabajo ha sido instalado en memoria de sólo lectura, ROM):\\
	
	El requerimiento de proporcionar Información de Instalación no incluye el requerimiento de continuar proporcionando servicio de soporte, garantía, o actualizaciones para un trabajo que haya sido modificado o instalado por el destinatario, o para el Producto de Usuario en el que se haya modificado o instalado.\\
	
	El acceso a la red puede ser denegado cuando la propia modificación afecte material y adversamente a la operación de la red o viole las reglas y protocolos de comunicación en la red. Las Fuentes Correspondientes transmitidas, y las Instrucciones de Instalación proporcionadas de acuerdo con esta sección, deben figurar en un formato documentado públicamente (y con una implementación disponible para el público en código fuente), y no debe necesitar claves de acceso especiales para la descompresión, lectura o copia.\\
	
	\item Condiciones adicionales.\\
	
	"Permisos Adicionales" son condicionantes que amplían los términos de esta Licencia permitiendo excepciones a una o más de sus condiciones. Los Permisos Adicionales que son aplicables al Programa completo deberán ser tratados como si estuviesen incluidos en esta Licencia, hasta los límites de validez impuestos por las leyes aplicables. Si los permisos adicionales se aplicasen sólo a una parte del Programa, esa parte podría ser usada de forma independiente bajo esos permisos, pero el Programa completo seguiría estando afectado por esta Licencia con independencia de los permisos adicionales.\\
	
	Cuando transmite una copia de un trabajo amparado, puede opcionalmente eliminar cualquier permiso adicional de esa copia, o de alguna parte del mismo. (Los permisos adicionales pueden haber establecido que sea requerida su eliminación en ciertos supuestos si usted modifica el trabajo.) Usted puede establecer permisos adicionales en material añadido por usted a un trabajo amparado, sobre el cual tiene o puede aportar sus permisos de copyright correspondientes. Con independencia de cualquier otra estipulación en esta Licencia, usted puede, para el material que añada a un trabajo amparado, (si está autorizado por los poseedores de copyright de ese material) añadir condiciones a esta Licencia con los siguientes términos:\\
	
	\begin{enumerate}
		
		\item Ausencia de garantía o limitación de responsabilidad diferente a los términos de las secciones 15 y 16 de esta Licencia; u\\
		
		\item Obligación de mantener determinados avisos legales razonables o atribuciones de autoría en el material o en los Avisos Legales Correspondientes mostrados por los trabajos que lo contengan; o\\
		
		\item Prohibir la tergiversación del origen del material, o solicitar que las diferencias respecto a la versión original sean advertidas de forma apropiada en las versiones modificadas del material; o\\
		
		\item Limitar la utilización de los nombres de los autores o licenciatarios del material con fines divulgativos ; o\\
		
		\item Negarse a ofrecer derechos afectados por leyes de registro para el uso de marcas empresariales, registradas o de servicio; o\\
		
		\item Requerir indemnización a los autores y poseedores de licencia de ese material, por parte de cualquier persona que transmite el material (o versiones modificadas del mismo) estableciendo obligaciones contractuales de responsabilidad sobre el destinatario, para cualquier responsabilidad que estas obligaciones contractuales impongan directamente sobre los autores y poseedores de licencia.\\
		
		\end {enumerate}
		
		Cualquiera otras condiciones adicionales no-permisivas son consideradas "restricciones posteriores" en el contexto de la sección 10. Si el Programa, tal cual lo recibió, o cualquier parte del mismo, contienen un aviso indicando que está amparado por esta Licencia junto a una cláusula de restricción posterior, usted puede suprimir esa cláusula. Si un documento de licencia contiene una restricción posterior pero permite modificar la licencia o el transmisión bajo esta Licencia, usted puede añadirla al material de un trabajo amparado por los términos de ese documento de licencia, siempre que la restricción posterior no se mantenga tras la modificación de la licencia o el transmisión.\\
		
		Si añade condiciones acordes a esta sección para un trabajo amparado, usted debe ubicar, en los archivos fuente involucrados, una declaración de los términos adicionales aplicables a esos archivos, o un aviso indicando dónde localizar los términos aplicables.\\
		
		Condiciones adicionales, permisivas o no, deben aparecer por escrito como licencias separadas, o figurar como excepciones; de todas formas los requisitos anteriores siempre son aplicables.\\
		
		\item Cancelación.\\
		
		Usted no puede transmitir o modificar un trabajo amparado salvo como expresamente se ha previsto en esta Licencia. Cualquier intento diferente de transmisión o modificación será considerado nulo, y automáticamente cancelará sus derechos respecto a esta Licencia (incluyendo cualquier patente conseguida según el párrafo tercero de la sección 11).\\
		
		Sin embargo, si deja de violar esta Licencia, entonces su licencia desde el poseedor del copyright correspondiente será restituida (a) provisionalmente, a menos que y hasta que el poseedor del copyright de por terminada explícita y permanentemente su licencia, y (b) permanentemente, si el poseedor del copyright no le ha notificado por algún cauce de la violación en los 60 días posteriores al cese. Además, su licencia desde el poseedor del copyright correspondiente será restituida permanentemente si el poseedor del copyright le notifica de la violación por algún cauce, es la primera vez que recibe la notificación de violación de esta Licencia (para cualquier trabajo) de ese poseedor de copyright, y usted subsana la violación antes de 30 días desde la recepción del aviso.\\
		
		La cancelación de sus derechos según esta sección no da por canceladas las licencias de terceros que hayan recibido copias o derechos a través de usted con esta Licencia. Si sus derechos han finalizado y no han sido restituidos de forma permanente, usted no está capacitado para recibir nuevas licencias para el mismo material según la sección 10.\\
		
		\item Aceptación no necesaria por tenencia de copias.\\
		
		No es necesario aceptar esta Licencia para recibir o ejecutar una copia del Programa. La distribución de un trabajo amparado surgida simplemente como consecuencia de usar transmisión entre usuarios para obtener una copia tampoco requiere aceptación. De todas formas, no puede usar otra salvo esta Licencia para transmitir o modificar cualquier trabajo amparado.\\
		
		\item Transmisión automática de licencia para destinatarios Cada vez que transmite un trabajo amparado, el destinatario recibirá automáticamente una licencia desde los poseedores originales, para ejecutar, modificar y transmitir ese trabajo, objeto de esta Licencia. Usted no es responsable de asegurar el cumplimiento por terceros de esta Licencia.\\
		
		Una "transacción de entidad" es una transacción que transfiere el control de una organización, o todos los bienes sustanciales de una, o subdivide una organización, o fusiona organizaciones. Si el transmisión de un trabajo amparado surge de una transacción de entidad, cada parte involucrada en esa transacción que reciba una copia del trabajo, también recibe cualquier licencia existente del trabajo cuya parte interesada tuviese o pudiese ofrecer según el párrafo anterior, además del derecho a tomar posesión de las Fuentes Correspondientes del trabajo a través de la parte interesada, si está en poder de dicha parte o se puede conseguir con un esfuerzo razonable.\\
		
		Usted no puede imponer restricciones posteriores en el ejercicio de los derechos otorgados o concedidos por esta Licencia. Por ejemplo, usted no puede imponer a la licencia pagos, derechos o otros cargos por el ejercicio de los derechos otorgados según esta Licencia, y usted no puede iniciar litigios (incluyendo demandas o contra demandas en pleitos) alegando que se infringen patentes por cambiar, usar, vender, ofrecer en alquiler o importar el Programa, o cualquier parte del mismo.\\
		
		\item Patentes.\\
		
		Un "colaborador" es un poseedor de copyright que autoriza el uso bajo esta Licencia del Programa o un trabajo en el que se base el Programa. El trabajo con esta licencia se denomina "versión en colaboración" con el colaborador. Todas las reivindicaciones de patentes en posesión o controladas por el colaborador se denominan "demandas de patente original", ya sean existentes o adquiridas, que hayan sido infringidas de alguna forma permitida por esta Licencia, al hacer, usar o vender la versión en colaboración, pero sin incluir demandas que sólo sean infracciones como consecuencia de modificaciones posteriores de la versión en colaboración. Para aclarar esta definición, "control" incluye el derecho de conceder sublicencias de patente según los requerimientos de esta Licencia.\\
		
		Cada colaborador le concede a usted una licencia de la patente no-exclusiva, global y libre de derechos bajo las reivindicaciones de patente de origen del colaborador, para el uso, modificación, venta, ofertas de venta, importación y otras formas de ejecución, modificación y redistribución del contenido de la versión en colaboración.\\
		
		En los siguientes tres párrafos, una "licencia de patente" se refiere a cualquier expresión de acuerdo o compromiso, independientemente de la denominación, que no imponga una patente (como puede ser el permiso expreso para ejecutar una patente o acuerdos para no imponer demandas por infracción de patente). "Conceder" estas licencias de patente a un tercero significa llegar a tal tipo de acuerdo o compromiso que no imponga una patente al tercero.\\
		
		Si usted transmite un trabajo amparado, conociendo que está afectado por licencia de patente, y no están disponibles de forma pública para su copia las Fuentes Correspondientes, sin cargo alguno y bajo los términos de esta Licencia, ya sea a través de un servidor de red público o mediante cualquier otro medio, entonces usted debe o (1) permitir que sean públicas las Fuentes Correspondientes, o (2) tratar de eliminar los beneficios de la licencia de patente para este trabajo en particular, o (3) tratar de extender, de forma adecuada a los requisitos de esta Licencia, la licencia de patente a terceros. "Conocer que está afectado" significa que usted tiene conocimiento actual de que, para la licencia de patente, el transmisión del trabajo amparado en un determinado país, o el uso del trabajo amparado por sus destinatarios en un determinado país, infringiría una o más patentes existentes en ese país que usted considera aplicables por algún motivo. Si para conseguir una transacción o acuerdo(o en un proceso relacionado con ellos), usted transmite o distribuye con fines de transmisión , un trabajo amparado, concediendo una licencia de patente para algún tercero que reciba el trabajo amparado, y autorizándole a usar, distribuir, modificar o transmitir una copia específica del trabajo amparado, entonces la licencia de patente que usted otorgue se extiende automáticamente a todos los receptores del trabajo amparado y cualquier trabajo basado en el mismo.\\
		
		Una licencia de patente es "discriminatoria" si no incluye dentro de su ámbito de cobertura, prohíbe el ejercicio, o está condicionada a no ejercitar uno o más de los derechos que están específicamente otorgados por esta Licencia. Usted no debe transmitir un trabajo amparado si está implicado en un acuerdo con terceros que estén relacionados con el negocio de la distribución de software, en el que usted haga pagos relacionados con su actividad del transmisión del trabajo, y donde se otorgue, a cualquier receptor del trabajo amparado, una licencia de patente discriminatoria (a) en relación con las copias de trabajo amparado transmitido por usted (o copias hechas desde éstas), o (b) directa o indirectamente relacionadas con productos específicos o paquetes que contengan el trabajo amparado, a menos que usted forme parte del acuerdo, o esa licencia de patente fuese otorgada antes del 28 de marzo de 2007.\\
		
		Ninguna disposición de esta Licencia debe ser considerada como excluyente o limitante de la aplicación de cualquier otra licencia o defensas legales contra la violación de las leyes de propiedad intelectual a que pudiera tener derecho bajo la ley de propiedad intelectual vigente.\\
		
		\item No condicionamiento de la libertad de terceros.\\
		
		Si a usted le son aplicables condiciones que contradicen las condiciones de esta Licencia (ya sea por orden judicial, acuerdo u otros), no queda eximido de cumplir las condiciones de esta Licencia. Si usted no puede transmitir un trabajo amparado cumpliendo simultáneamente sus obligaciones con esta Licencia y con cualquier otra pertinente, entonces no podrá transmitirlo de ninguna forma. Por ejemplo, si usted se compromete con términos que le obligan a obtener derechos por el transmisión a terceros, la única forma de satisfacer ambos condicionantes y esta Licencia es abstenerse completamente de transmitir el Programa.\\
		
		\item Uso conjunto con la Licencia Pública General Affero GNU.\\
		
		Con independencia de cualquier disposición en esta Licencia, usted tiene permiso para enlazar o combinar cualquier trabajo amparado con otro trabajo amparado por la versión 3 de la Licencia Pública General Affero GNU, para formar un solo trabajo combinado, y transmitir el trabajo resultante. Los términos de esta Licencia seguirán siendo aplicables a la parte formada por el trabajo amparado, pero los condicionantes especiales de la Licencia Pública General Affero GNU, en su sección 13, relativos a la interacción mediante redes, serán aplicables a la combinación de ambas partes.\\
		
		\item Versiones Revisadas de esta Licencia.\\
		
		La Fundación para el Software Libre puede publicar revisiones y/o nuevas versiones de la Licencia Pública General GNU de vez en cuando. Esas versiones serán similares en espíritu a la versión actual, pero podrán diferir en algunos detalles para afrontar nuevos problemas o situaciones.\\
		
		A cada versión se le da un número distintivo. Si el Programa especifica que le es aplicable cierto número de versión de la Licencia Pública General o "cualquier versión posterior", usted tiene la posibilidad de adoptar los términos y condiciones de la versión indicada o de cualquier otra versión posterior publicada por la Fundación para el Software Libre. Si el Programa no especifica un número de versión de la Licencia Pública General, usted puede elegir cualquier versión que haya sido publicada por la Fundación para el Software Libre.\\
		
		Si el Programa especifica que un apoderado puede decidir qué versiones de la Licencia Pública General pueden aplicarse en el futuro, la declaración pública de aceptación que el apoderado haga de una versión le autoriza a usted con carácter permanente a elegir esa versión para el Programa.\\
		
		Versiones de licencia posteriores pueden otorgarle permisos adicionales o diferentes.\\
		
		De todas formas, no pueden imponerse obligaciones adicionales a cualquier autor o poseedor de copyright como consecuencia de su elección de adoptar una versión posterior.\\
		
		\item Ausencia de Garantía.\\
		
		EL PROGRAMA NO TIENE GARANTÍA ALGUNA, HASTA LOS LÍMITES PERMITIDOS POR LAS LEYES APLICABLES SALVO CUANDO SE ESTABLEZCA LO CONTRARIO POR ESCRITO EL POSEEDOR DEL COPYRIGHT Y/O TERCEROS PROPORCIONAN EL PROGRAMA "TAL CUAL" SIN GARANTÍA DE NINGÚN TIPO, YA SEA EXPLÍCITA O IMPLÍCITA, INCLUYENDO, PERO SIN LIMITARSE A, LAS GARANTÍAS IMPLÍCITAS MERCANTILES Y DE APTITUD PARA UN PROPÓSITO DETERMINADO. USTED ASUME CUALQUIER RIESGO RELATIVO A LA CALIDAD Y RENDIMIENTO DEL PROGRAMA. SI EL PROGRAMA FUESE DEFECTUOSO, USTED ASUME EL COSTE DE CUALQUIER COSTE DE SERVICIO, REPARACIÓN O CORRECCIÓN.\\
		
		\item Limitación de Responsabilidad\\
		
		EN NINGÚN CASO, SALVO REQUERIMIENTO POR LEYES APLICABLES O ACUERDO POR ESCRITO, PODRÁ UN POSEEDOR DE COPYRIGHT, O UN TERCERO QUE MODIFIQUE O TRANSMITE EL PROGRAMA SEGÚN LO INDICADO ANTES, HACERLE A USTED RESPONSABLE DE DAÑO ALGUNO, INCLUYENDO CUALQUIER DAÑO GENERAL, ESPECIAL, OCASIONAL O DERIVADO QUE SURJA DEL USO O LA INCAPACIDAD DE USO DEL PROGRAMA (INCLUYENDO PERO SIN LIMITARSE A LA PÉRDIDA DE DATOS O LA PRESENTACIÓN NO PRECISA DE DATOS O PÉRDIDAS SUFRIDAS POR USTED O TERCEROS O FALLO DEL PROGRAMA AL INTERACTUAR CON OTROS PROGRAMAS), INCLUSO SI EL POSEEDOR O UN TERCERO HA SIDO ADVERTIDO DE LA POSIBILIDAD DE TALES DAÑOS.\\
		
		\item Interpretación de la sección "Ausencia de Garantía".\\
		
		Si la ausencia de garantía y la limitación de responsabilidad descrita anteriormente no tuviesen efecto legal a nivel local en todos sus términos, los juzgados aplicarán las leyes locales que más se aproximen a la exención de responsabilidad civil en lo relativo al Programa, a menos que la copia del Programa esté acompañada mediante pago de una garantía o compromiso de responsabilidad.\\
		
	\end{enumerate}
	
