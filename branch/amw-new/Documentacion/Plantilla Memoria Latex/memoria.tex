% ------------------------------------------------------------------------------
% Este fichero es parte de la plantilla LaTeX para la realización de Proyectos
% Final de Grado, protegido bajo los términos de la licencia GFDL.
% Para más información, la licencia completa viene incluida en el
% fichero fdl-1.3.tex

% Copyright (C) 2012 SPI-FM. Universidad de Cádiz
% ------------------------------------------------------------------------------

\documentclass[a4paper,11pt]{book}

% PAQUETES
\usepackage{./estilo/paquetes}
\usepackage{./estilo/colores}
\usepackage{./estilo/comandos}

% Ruta al directorio de imágenes
\graphicspath{{./img/}} 

% METADATOS
\title{AssessMediaWiki}
\author{Jose Alberto Garcia Pinteño}
\date{Septiembre 2015} 
 
\begin{document}

\pagestyle{plain}

% PORTADAS
% ------------------------------------------------------------------------------
% Este fichero es parte de la plantilla LaTeX para la realización de Proyectos
% Final de Grado, protegido bajo los términos de la licencia GFDL.
% Para más información, la licencia completa viene incluida en el
% fichero fdl-1.3.tex

% Copyright (C) 2012 SPI-FM. Universidad de Cádiz
% ------------------------------------------------------------------------------


\begin{titlepage}

  \begin{center}

    \includegraphics[width=0.3\textwidth]{logo-uca.png} \\
    
    \vspace{2.5cm}
    
    \LARGE{\textbf{ESCUELA SUPERIOR DE INGENIERÍA}} \\
    
    \vspace{1.0cm}
    
    \Large{\textbf{INGENIERÍA INFORMÁTICA DE SISTEMAS}} \\
    
    \vspace{3.0cm}
    
    \Large{PROYECTO DE FIN DE CARRERA} \\
    
    \vspace{2.5cm}
    
    \Large{José Alberto García Pinteño} \\
  
    \vspace{0.5cm}

    \large{\today}
    
  \end{center}
\end{titlepage}

\cleardoublepage

\input{./portadas/portada-secundaria}
\cleardoublepage

% PRELIMINARES
% ------------------------------------------------------------------------------
% Este fichero es parte de la plantilla LaTeX para la realización de Proyectos
% Final de Grado, protegido bajo los términos de la licencia GFDL.
% Para más información, la licencia completa viene incluida en el
% fichero fdl-1.3.tex

% Copyright (C) 2012 SPI-FM. Universidad de Cádiz
% ------------------------------------------------------------------------------

\thispagestyle{empty}

\noindent \textbf{\begin{Large}\textit{Agradecimientos}\end{Large}} 
\newline
\newline
\noindent\textit{Me gustaria darle las gracias a mis padres, ya que sin ellos no podria haber llegado hasta aqui, a mi familia por todo su apoyo y a todos los profesores que se han esforzado por dejar un poco de su conocimiento en mi, en especial a Manuel Palomo, por toda su paciencia y apoyo.}

\newpage

% ------------------------------------------------------------------------------
% Este fichero es parte de la plantilla LaTeX para la realización de Proyectos
% Final de Grado, protegido bajo los términos de la licencia GFDL.
% Para más información, la licencia completa viene incluida en el
% fichero fdl-1.3.tex

% Copyright (C) 2012 SPI-FM. Universidad de Cádiz
% ------------------------------------------------------------------------------

\thispagestyle{empty}

\noindent \textbf{\begin{Large}Resumen\end{Large}} 
\newline
\newline
\noindent

Wiki (del hawaiano wiki, "rápido") es el nombre que recibe un sitio web cuyas páginas pueden ser editadas directamente desde el navegador, donde los usuarios crean, modifican o eliminan contenidos que, generalmente, comparten.
\newline

En este proyecto usaremos los wikis como herramienta docente, permitiendo que los alumnos trabajen sobre el y posteriormente se evalúen entre ellos. Este uso ya esta en practica actualmente, y fue premiado entre los mejores Proyectos de Innovación y Mejora Docente y Proyectos de Innovación Educativa que finalizaron durante el curso 2011/2012. \cite{Balderas:2012}
\newline

 Los wikis son un sistema muy popular como ayuda a la docencia. Cuando el número de estudiantes y la cantidad de información almacenada en un wiki aumentan, evaluar el trabajo de cada estudiante resulta difícil. Los wikis mantienen un registro con las diferencias entre las revisiones consecutivas de los artículos, que pueden ser usadas para la evaluación del aprendizaje. Esta información puede evaluarse a lo largo de la vida del wiki para obtener datos sobre la actividad de los estudiantes.
\newline

AssessMediaWiki es una aplicación web de código abierto que, al conectarse a una instalación MediaWiki, proporciona procedimientos de autoevaluación, hetero evaluación y evaluación entre iguales, a la vez que mantiene información sobre esas evaluaciones. Los supervisores pueden obtener informes que ayudan en la evaluación de los estudiantes.
\newline

Aunque hay un gran número de extensiones para el sistema MediaWiki, no hemos encontrado ninguna que permitiera evaluar contribuciones individuales a un wiki. La mayoría de las aproximaciones solo ofrecen formas de evaluar una versión en particular de un artículo (normalmente la más reciente), siendo ineficaces en este caso. Por ello, para evaluar la calidad de las contribuciones creamos AssessMediaWiki.
\newline

AssessMediaWiki implementa como base dos roles de usuario distintos: supervisores y estudiantes. Los estudiantes pueden elegir entre distintas opciones: evaluar una revisión, comprobar sus propias aportaciones evaluadas y verificar las evaluaciones ya enviadas. Por otro lado, los supervisores tienen un mayor número de opciones, como modificar los parámetros de los programas o vigilar las evaluaciones que los alumnos vayan haciendo.
\newline

\noindent {\bf Palabras clave:} AssessMediaWiki, MediaWiki, Wiki, software libre, evaluación

\newpage


\frontmatter

% INDICES
\tableofcontents
\listoffigures
\listoftables

\mainmatter

\chapter{Introducción}
% ------------------------------------------------------------------------------
% Este fichero es parte de la plantilla LaTeX para la realización de Proyectos
% Final de Grado, protegido bajo los términos de la licencia GFDL.
% Para más información, la licencia completa viene incluida en el
% fichero fdl-1.3.tex

% Copyright (C) 2012 SPI-FM. Universidad de Cádiz
% ------------------------------------------------------------------------------

\section{Motivación}
Tras trabajar durante un breve periodo de tiempo en AssessMediaWiki gracias a la beca Icaro, se planteo la posibilidad de seguir trabajando sobre el mismo, pero esta vez para usarlo como proyecto de fin de carrera.
\newline

Tras esa experiencia, viendo todo lo que había prendido de ella (manejar nuevos lenguajes y herramientas de desarrollo, distintas formas de enfocar los problemas, etc), mi interés por aprender y explorar diversos métodos didácticos y el interés y respeto que tengo hacia el software libre y su comunidad, decidí continuar desarrollando AssessMediaWiki y aprovechar para usarlo como proyecto de fin de carrera, con la posterior motivación extra de intentar hacer publicaciones con el.

\section{Alcance} 

Este proyecto está orientado a añadir nuevas herramientas y funciones a AssessMediaWiki, dando un abanico mas amplio de posibilidades a los docentes a la hora de interactuar con los alumnos.

\section{Glosario de Términos} 

\begin{itemize}
	\item AMW - AssessMediaWiki
	\item Edición - Aportación realizada por un alumno al MediaWiki
	\item Metaevaluación - evaluación realizada sobre una evaluación existente, para poder ver así la corrección de la evaluación existente
\end{itemize}

\section{Organización del documento}

En el documento se presenta primero la planificación del desarrollo del sistema, así como la metodología de trabajo a seguir.
\newline

En el apartado de desarrollo se muestran un análisis de requisitos y de el sistema actual. También se mesta el proceso para la implementación de las nuevas funciones, así como las modificaciones en las bases de datos.
\newline

En la parte final del documento encontramos el epilogo con el manual de instalación, el manual de uso y la bibliografía utilizada en este documento.





\chapter{Planificación}
% ------------------------------------------------------------------------------
% Este fichero es parte de la plantilla LaTeX para la realización de Proyectos
% Final de Grado, protegido bajo los términos de la licencia GFDL.
% Para más información, la licencia completa viene incluida en el
% fichero fdl-1.3.tex

% Copyright (C) 2012 SPI-FM. Universidad de Cádiz
% ------------------------------------------------------------------------------


\section{Metodología de desarrollo}
Definición del proceso de desarrollo, ciclo de vida y metodología empleada durante la elaboración del proyecto. Las fases y/o iteraciones que proponga el método empleado deberán quedar recogidas en la planificación que se detalle más adelante.

Se ha optado por una metodología de desarrollo en cascada, como se aprecia en la siguiente imagen:

\begin{figure}[h!]
	\centering
	\includegraphics[width=1.0\textwidth]{Cascada.jpg}
	\caption{Desarrollo en cascada.}
\end{figure}

\section{Planificación del proyecto}
Estimación temporal y definición del calendario básico (hitos principales e iteraciones). Desarrollo de la planificación detallada, utilizando un diagrama de Gantt.\\

Se debe incluir una comparación cuantitativa del tiempo y el esfuerzo realmente invertido frente al estimado y planificado. Estos datos pueden recogerse del sistema de gestión de tareas empleado para el seguimiento del proyecto.


\begin{figure}[h]
	\centering
	\includegraphics[width=1.0\textwidth]{Gantt-numerico.jpg}
	\caption{Datos del diagrama de Gantt.}
\end{figure}

\begin{figure}[h!]
	\centering
	\includegraphics[width=1.0\textwidth]{Gantt-grafica.jpg}
	\caption{Diagrama de Gantt.}
\end{figure}


\begin{figure}[h!]
	\centering
	\includegraphics[width=0.6\textwidth]{Gantt-grafica-girada.jpg}
	\caption{Diagrama de Gantt (girada).}
\end{figure}

\clearpage

\section{Organización}
Relación de las personas (roles) involucradas en el proyecto y de cómo se estructuran las relaciones entre las mismas para ejecutar el proyecto. Relación de los recursos inventariables utilizados en el proyecto: equipamiento informático (hardware y software), herramientas empleadas, etc. \\

Para llevar a cabo este proyecto es necesario un MediaWiki. un servidor para alojar AssessMediaWiki y en caso de que el profesor no posea los conocimientos necesarios para la instalación, la colaboración de personal que facilite la instalación de AssessMediaWiki

\section{Riesgos}
Enumeración de los riesgos del proyecto, indicando su posible impacto (efecto que la ocurrencia del citado riesgo tendría en el desarrollo del proyecto) y la probabilidad de ocurrencia. Una vez los riesgos son identificados y priorizados, hay que definir los planes necesarios para reducir los efectos del riesgo una vez se haya materializado o disminuir que este ocurra.\\

De momento hay un pequeño problema con los nombres de los roles creados en el sistema AssessMediaWiki y los ejercicios de evaluación, si los nombres llevan espacio no se podrán eliminar, y es posible que no puedan ser modificados, es un pequeño detalle que se puede arreglar en un trabajo futuro, para evitar este problema basta con usar barras bajas en lugar de espacios.\\

También puede considerarse un riesgo usar el framework (\href{http://www.codeigniter.com/}{CodeIgniter}) sin conocimiento previo, así como que en un futuro dicho framework deje de recibir soporte.


% DESARROLLO
\part{Desarrollo}

\chapter{Requisitos del Sistema}
% ------------------------------------------------------------------------------
% Este fichero es parte de la plantilla LaTeX para la realización de Proyectos
% Final de Grado, protegido bajo los términos de la licencia GFDL.
% Para más información, la licencia completa viene incluida en el
% fichero fdl-1.3.tex

% Copyright (C) 2012 SPI-FM. Universidad de Cádiz
% ------------------------------------------------------------------------------


\section{Situación actual} 
En la versión actual (la que está funcionando en estos momentos) de AssessMediaWiki nos encontramos con un algoritmo de selección que escoge una edición de forma aleatoria entre las $n$ ediciones mas significativas (las de mayor tamaño), de forma que en el peor de los casos nos encontramos con la situación de que las $n-1$ ediciones mas significativas nunca serán seleccionadas.
\newline
Esta versión de AssessMediaWiki también carece de un procesado de ediciones para ciertos wiki-comportamientos (ráfagas, correcciones ortográficas, etc) y los roles de los usuarios se limitan a los de alumno y supervisor.\\
Estos situaciones han sido complementadas con esta nueva versión, añadiendo un algoritmo distinto de selección de ediciones, un sistema de administración de roles y el procesado de ráfagas

\subsection{Entorno Tecnológico}
Para este proyecto sera necesario contar con un MediaWiki, normalmente alojado en un servidor, en el cual seria recomendable alojar el mismo software del proyecto.

\subsection{Fortalezas y Debilidades}
Fortalezas: Herramienta muy útil para añadir nuevas posibilidades a los métodos docentes.//

Debilidades: Es necesario un mínimo de conocimientos para su instalación, al igual que para la instalación del MediaWiki.

\section{Objetivos del Sistema}
El objetivo del sistema es añadir nuevas funciones al método docente y darle a los alumnos mas responsabilidades, así como analizar su participación y resultados.

\section{Catálogo de Requisitos}

\subsection{Requisitos funcionales}
Listado de las nuevas funcionalidades que ofrece el sistema:

\begin{itemize}
	\item Creación y edición de nuevos roles.
	\item Creación y edición de ejercicios de evaluación.
	\item Procesado de ráfagas.
	\item Preasignaciones de ediciones.
\end{itemize}

\subsection{Requisitos no funcionales}
Descripción de otros requisitos (relacionados con la calidad del software) que el sistema deberá satisfacer: portabilidad, seguridad, estándares de obligado cumplimiento, accesibilidad, usabilidad, etc:
\begin{itemize}
	\item MediaWiki.
	\item Servidor para hospedar tanto el MediaWiki como AssessMediaWiki.
	\item Conexión a internet.
\end{itemize}

\subsection{Requisitos de información}
En esta sección se describen los requisitos de gestión de información (datos) que el sistema debe gestionar:
\begin{itemize}
	\item IDs de los usuarios del MediaWiki.
	\item IDs de las ediciones creadas por los usuarios.
	\item Notas y comentarios asignados a ediciones evaluadas.
	\item Parámetros de configuración
	\item Roles creados por el administrador.
	\item Ejercicios de evaluación y sus configuraciones.
\end{itemize}

\section{Alternativas de Solución}
Se presento la posibilidad de usar \href{http://evalcomix.uca.es/}{evalcomix} \cite{Evalcomix} como herramienta de evaluación alternativa.

\section{Solución Propuesta}
Se consideró mejor opción implementar un método propio que fuese escalable, por lo que se descarto usar \href{http://evalcomix.uca.es/}{evalcomix} \cite{Evalcomix} en el desarrollo del sistema.
\newline

Se ha implementado un sistema de evaluación en el que el docente puede configurar cuantas ediciones se evaluaran de cada estudiante y cuantas evaluaciones recibirá cada edición.

\chapter{Análisis del Sistema}
% ------------------------------------------------------------------------------
% Este fichero es parte de la plantilla LaTeX para la realización de Proyectos
% Final de Grado, protegido bajo los términos de la licencia GFDL.
% Para más información, la licencia completa viene incluida en el
% fichero fdl-1.3.tex

% Copyright (C) 2012 SPI-FM. Universidad de Cádiz
% ------------------------------------------------------------------------------


\section{Modelo Conceptual}

A partir de los requisitos de información, se desarrollará un diagrama conceptual de clases UML, identificando las clases, atributos, relaciones, restricciones adicionales y reglas de derivación necesarias.

\begin{figure}
	\centering
	\includegraphics[width=0.6\textwidth]{db1girada.jpg}
	\caption{Diagrama de la base de datos de AMW 1.0.}
\end{figure}

\begin{figure}
	\centering
	\includegraphics[width=0.6\textwidth]{db2girada.jpg}
	\caption{Diagrama de la base de datos de AMW 2.0.}
\end{figure}

\newpage

\section{Modelo de Casos de Uso}

\begin{figure}
	\centering
	\includegraphics[width=1.0\textwidth]{Casos-de-uso.jpg}
	\caption{Modelo de casos de uso.}
\end{figure}

\subsection{Actores} 
Los actores básicos son:
\newline
- Administrador/Profesor: El usuario supervisor del sistema
\newline
- Estudiante: Usuario que realizara las evaluaciones de las ediciones suyas o de sus compañeros
\newline

Pero hay que tener en cuenta que el administrador puede crear nuevos roles, lo que generara nuevos actores dependiendo de los permisos que se les conceda a dichos roles.

\newpage

\subsection{Descripción de los casos de uso}
Caso de uso:  \textbf{Administrar roles}
\newline
Descripción: El usuario se dirige a la pagina de administración de roles.
\newline
Actores: Usuario, Sistema
\newline
Precondiciones: El usuario esta logeado y es administrador.
\newline
Postcondiciones: El sistema muestra la pagina de administración de roles.
\newline
Escenario principal:
\begin{itemize}
	\item 1. El usuario hace clic en el enlace a la administración de roles.
	\item El sistema muestra la pagina de administración de roles.
\end{itemize}

Escenario alternativo 1: 
\begin{itemize}
	\item *.* En cualquier momento del caso de uso el usuario usa los botones de navegación (atrás, adelante, actualizar) y termina el proceso actual sin guardar los datos ni ejecutarse ninguna acción.
\end{itemize}
Escenario alternativo 2:
\begin{itemize}
\item *.* En cualquier momento del caso de uso el usuario viaja a otra sección de la web o a otra web y termina el proceso actual sin guardar los datos ni ejecutarse ninguna acción.
\end{itemize}

Caso de uso: \textbf{Crear rol}
\newline
Descripción: El usuario decide crear un nuevo rol.
\newline
Actores: Usuario, Sistema
\newline
Precondiciones: El usuario esta logeado y es administrador y se encuentra en la pagina de administración de roles.
\newline
Postcondiciones: Se introduce un nuevo rol en la base de datos del sistema.
\newline
Escenario principal:
\begin{itemize}
	\item 1. El usuario rellena los datos para crear un nuevo rol en la pagina de administración de roles.
	\item 2. El usuario hace clic en crear rol.
	\item 3. El sistema comprueba que no existe ningún otro rol en el sistema con el nombre del nuevo rol a crear.
	\item 4. El sistema añade el nuevo rol a la base de datos del sistema.
 \item 5. El sistema actualiza la pagina de administración de roles, mostrando también el nuevo rol añadido.
\end{itemize}
Escenario alternativo 1: 
\begin{itemize}
	\item *.* En cualquier momento del caso de uso el usuario usa los botones de navegación (atrás, adelante, actualizar) y termina el proceso actual sin guardar los datos ni ejecutarse ninguna acción.
\end{itemize}
Escenario alternativo 2:
\begin{itemize}
	\item *.* En cualquier momento del caso de uso el usuario viaja a otra sección de la web o a otra web y termina el proceso actual sin guardar los datos ni ejecutarse ninguna acción.
\end{itemize}
Escenario alternativo 3:
\begin{itemize}
	\item 3.a El sistema detecta que el nuevo rol a introducir ya existe en el sistema.
	\item 4.a El sistema actualiza la pagina de administración de roles, sin añadir el nuevo rol que el usuario pretendía añadir.
\end{itemize}

Caso de uso: \textbf{Editar permisos de rol}
\newline
Descripción: El usuario decide editar los permisos de un rol.
\newline
Actores: Usuario, Sistema
\newline
Precondiciones: El usuario esta logeado y es administrador y se encuentra en la pagina de administración de roles.
\newline
Postcondiciones: Se modifican los permisos de un rol en la base de datos del sistema.
\newline
Escenario principal:
\begin{itemize}
\item 1. El usuario rellena los datos para modificar los permisos de un rol en la pagina de administración de roles.
\item 2. El usuario hace clic en modificar rol.
\item 3. El sistema modifica los permisos del rol en la base de datos del sistema.
\item 4. El sistema actualiza la pagina de administración de roles, mostrando los cambios efectuados.
\end{itemize}
Escenario alternativo 1: 
\begin{itemize}
	\item *.* En cualquier momento del caso de uso el usuario usa los botones de navegación (atrás, adelante, actualizar) y termina el proceso actual sin guardar los datos ni ejecutarse ninguna acción.
\end{itemize}
Escenario alternativo 2:
\begin{itemize}
	\item *.* En cualquier momento del caso de uso el usuario viaja a otra sección de la web o a otra web y termina el proceso actual sin guardar los datos ni ejecutarse ninguna acción.
\end{itemize}

Caso de uso: \textbf{Eliminar rol}
\newline
Descripción: El usuario decide eliminar rol.
\newline
Actores: Usuario, Sistema
\newline
Precondiciones: El usuario esta logeado y es administrador y se encuentra en la pagina de administración de roles.
\newline
Postcondiciones: Se elimina el rol seleccionado de la base de datos del sistema.
\newline
Escenario principal:
\begin{itemize}
	\item 1. El sistema muestra una lista desplegable con los roles existentes en el sistema, exceptuando los de administrador y estudiante.
	\item 2. El usuario selecciona el rol que desea eliminar.
	\item 3. El usuario hace clic en el botón de eliminar rol.
	\item 4. El sistema elimina todas las asignaciones de usuarios al rol a eliminar.
	\item 5. El sistema elimina el rol seleccionado de la base de datos del sistema.
	\item 6. El sistema actualiza la pagina de administración de roles, mostrando los cambios efectuados.
\end{itemize}
Escenario alternativo 1:
\begin{itemize} 
	\item *.* En cualquier momento del caso de uso el usuario usa los botones de navegación (atrás, adelante, actualizar) y termina el proceso actual sin guardar los datos ni ejecutarse ninguna acción.
\end{itemize}
Escenario alternativo 2:
\begin{itemize}
	\item *.* En cualquier momento del caso de uso el usuario viaja a otra sección de la web o a otra web y termina el proceso actual sin guardar los datos ni ejecutarse ninguna acción.
\end{itemize}

Caso de uso: \textbf{Asignar rol}
\newline
Descripción: El usuario decide asignar un rol a un estudiante.
\newline
Actores: Usuario, Sistema
\newline
Precondiciones: El usuario esta logeado y es administrador y se encuentra en la pagina de administración de roles.
\newline
Postcondiciones: Se introduce un nuevo rol en la base de datos del sistema.
\newline
Escenario principal:
\begin{itemize}
	\item 1. El sistema muestra una lista desplegable con los usuarios existentes en el sistema y otra con los roles existentes en el sistema.
	\item 2. El usuario selecciona un usuario y un rol de las listas.
	\item 3. El usuario hace clic en el botón de asignar rol.
	\item 4. El sistema comprueba que el usuario no tenia ningún rol asignado.
	\item 5. El sistema introduce los datos del usuario seleccionado y el rol que se le va a asignar.
	\item 6. El sistema actualiza la pagina de administración de roles, mostrando los cambios efectuados.
\end{itemize}
Escenario alternativo 1: 
\begin{itemize}
	\item *.* En cualquier momento del caso de uso el usuario usa los botones de navegación (atrás, adelante, actualizar) y termina el proceso actual sin guardar los datos ni ejecutarse ninguna acción.
\end{itemize}
Escenario alternativo 2:
\begin{itemize}
	\item *.* En cualquier momento del caso de uso el usuario viaja a otra sección de la web o a otra web y termina el proceso actual sin guardar los datos ni ejecutarse ninguna acción.
\end{itemize}
Escenario alternativo 3:
\begin{itemize}
	\item 4.a El sistema detecta que el usuario ya tiene un rol asignado.
	\item 5.a El sistema detecta que el nuevo rol que se le quiere asignar es el de estudiante.
	\item 7.a El sistema borra de la base de datos la asignación previa que tuviera el usuario seleccionado.
	\item 8.a El sistema actualiza la pagina de administración de roles, mostrando los cambios efectuados.
\end{itemize}
Escenario alternativo 4:
\begin{itemize}
	\item 5.b El sistema detecta que el nuevo rol que se quiere asignar no es el de estudiante.
	\item 6.b El sistema actualiza la base de datos con los nuevos datos introducidos
	\item 7.b El sistema actualiza la pagina de administración de roles mostrando los cambios efectuados.
\end{itemize}

Caso de uso: \textbf{Administrar ejercicios de evaluación}
\newline
Descripción: El usuario se dirige a la pagina de administración de ejercicios de evaluación
\newline
Actores: Usuario, Sistema
\newline
Precondiciones: El usuario esta logeado y es administrador.
\newline
Postcondiciones: El sistema muestra la pagina de administración de ejercicios de evaluación
\newline
Escenario principal:
\begin{itemize}
	\item 1. El usuario hace clic en el enlace a la administración de ejercicios de evaluación
	\item 2. El sistema muestra la pagina de administración de ejercicios de evaluación
\end{itemize}
Escenario alternativo 1: 
\begin{itemize}
	\item *.* En cualquier momento del caso de uso el usuario usa los botones de navegación (atrás, adelante, actualizar) y termina el proceso actual sin guardar los datos ni ejecutarse ninguna acción.
\end{itemize}
Escenario alternativo 2: 
\begin{itemize}
	\item *.* En cualquier momento del caso de uso el usuario viaja a otra sección de la web o a otra web y termina el proceso actual sin guardar los datos ni ejecutarse ninguna acción.
\end{itemize}

Caso de uso: \textbf{Crear ejercicio de evaluación}
\newline
Descripción: El usuario decide crear un nuevo ejercicio de evaluación
\newline
Actores: Usuario, Sistema
\newline
Precondiciones: El usuario esta logeado y es administrador y se encuentra en la pagina de administración de ejercicios de evaluación
\newline
Postcondiciones: Se introduce un nuevo ejercicio de evaluación en la base de datos del sistema.
\newline
Escenario principal:
\begin{itemize}
	\item 1. El usuario rellena los datos para crear un nuevo ejercicio de evaluación en la pagina de administración de ejercicios de evaluación
	\item 2. El usuario hace clic en crear ejercicio de evaluación
	\item 3. El sistema comprueba que no existe ningún otro ejercicio de evaluación en el sistema con el nombre del nuevo ejercicio de evaluación a crear.
	\item 4. El sistema añade el nuevo ejercicio de evaluación a la base de datos del sistema.
	\item 5. El sistema actualiza la pagina de administración de ejercicios de evaluación, mostrando también el nuevo ejercicio de evaluación añadido.
\end{itemize}
Escenario alternativo 1: 
\begin{itemize}
	\item *.* En cualquier momento del caso de uso el usuario usa los botones de navegación (atrás, adelante, actualizar) y termina el proceso actual sin guardar los datos ni ejecutarse ninguna acción.
\end{itemize}
Escenario alternativo 2:
\begin{itemize}
	\item *.* En cualquier momento del caso de uso el usuario viaja a otra sección de la web o a otra web y termina el proceso actual sin guardar los datos ni ejecutarse ninguna acción.
\end{itemize}
Escenario alternativo 3:
\begin{itemize}
	\item 3.a El sistema detecta que el nuevo ejercicio de evaluación a introducir ya existe en el sistema.
	\item 4.a El sistema actualiza la pagina de administración de ejercicios de evaluación, sin añadir el nuevo ejercicio de evaluación que el usuario pretendía añadir.
\end{itemize}

Caso de uso: \textbf{Editar ejercicio de evaluación}
\newline
Descripción: El usuario decide editar un ejercicio de evaluación
\newline
Actores: Usuario, Sistema
\newline
Precondiciones: El usuario esta logeado y es administrador y se encuentra en la pagina de administración de ejercicios de evaluación
\newline
Postcondiciones: Se modifican un ejercicio de evaluación en la base de datos del sistema.
\newline
Escenario principal:
\begin{itemize}
	\item 1. El usuario rellena los datos para modificar un ejercicio de evaluación en la pagina de administración de roles.
	\item 2. El usuario hace clic en modificar ejercicio de evaluación
	\item 4. El sistema modifica el ejercicio de evaluación en la base de datos del sistema.
	\item 5. El sistema actualiza la pagina de administración de ejercicios de evaluación, mostrando los cambios efectuados.
\end{itemize}
Escenario alternativo 1: 
\begin{itemize}
	\item *.* En cualquier momento del caso de uso el usuario usa los botones de navegación (atrás, adelante, actualizar) y termina el proceso actual sin guardar los datos ni ejecutarse ninguna acción.
\end{itemize}
Escenario alternativo 2:
\begin{itemize}
	\item *.* En cualquier momento del caso de uso el usuario viaja a otra sección de la web o a otra web y termina el proceso actual sin guardar los datos ni ejecutarse ninguna acción.
\end{itemize}

Caso de uso: \textbf{Eliminar ejercicio de evaluación}
\newline
Descripción: El usuario decide eliminar un ejercicio de evaluación
\newline
Actores: Usuario, Sistema
\newline
Precondiciones: El usuario esta logeado y es administrador y se encuentra en la pagina de administración de ejercicio de evaluación
\newline
Postcondiciones: Se elimina el ejercicio de evaluación seleccionado de la base de datos del sistema.
\newline
Escenario principal:
\begin{itemize}
	\item 1. El sistema muestra una lista desplegable con los ejercicios de evaluación existentes en el sistema.
	\item 2. El usuario selecciona el ejercicio de evaluación que desea eliminar.
	\item 3. El usuario hace clic en el botón de eliminar ejercicio de evaluación
	\item 4. El sistema elimina el ejercicio de evaluación seleccionado de la base de datos del sistema.
	\item 5. El sistema actualiza la pagina de administración de ejercicios de evaluación, mostrando los cambios efectuados.
\end{itemize}
Escenario alternativo 1: 
\begin{itemize}
	\item *.* En cualquier momento del caso de uso el usuario usa los botones de navegación (atrás, adelante, actualizar) y termina el proceso actual sin guardar los datos ni ejecutarse ninguna acción.
\end{itemize}
Escenario alternativo 2:
\begin{itemize}
	\item *.* En cualquier momento del caso de uso el usuario viaja a otra sección de la web o a otra web y termina el proceso actual sin guardar los datos ni ejecutarse ninguna acción.
\end{itemize}

Caso de uso: \textbf{Editar criterios de evaluación}
\newline
Descripción: El usuario decide editar los criterios de evaluación de un ejercicio de evaluación
\newline
Actores: Usuario, Sistema
\newline
Precondiciones: El usuario esta logeado y es administrador y se encuentra en la pagina de administración de ejercicio de evaluación
\newline
Postcondiciones: Se modifican los criterios de evaluación de un ejercicio de evaluación en la base de datos del sistema.
\newline
Escenario principal:
\begin{itemize}
	\item 1. El sistema muestra una tabla con los ejercicio de evaluación y criterios de evaluación existentes en el sistema.
	\item 2. El usuario modifica los criterios de evaluación de uno de los ejercicios de evaluación
	\item 3. El usuario hace clic en el botón de modificar criterios de evaluación
	\item 4. El sistema modifica los datos para el ejercicio de evaluación seleccionado.
	\item 5. El sistema actualiza la pagina de administración de roles, mostrando los cambios efectuados.
\end{itemize}
Escenario alternativo 1: 
\begin{itemize}
	\item *.* En cualquier momento del caso de uso el usuario usa los botones de navegación (atrás, adelante, actualizar) y termina el proceso actual sin guardar los datos ni ejecutarse ninguna acción.
\end{itemize}
Escenario alternativo 2:
\begin{itemize}
	\item *.* En cualquier momento del caso de uso el usuario viaja a otra sección de la web o a otra web y termina el proceso actual sin guardar los datos ni ejecutarse ninguna acción.
\end{itemize}

Caso de uso: \textbf{Hacer Preasignaciones}
\newline
Descripción: El usuario decide realizar las preasignaciones para un ejercicio de evaluación
\newline
Actores: Usuario, Sistema
\newline
Precondiciones: El usuario esta logeado y es administrador y se encuentra en la pagina de administración de ejercicio de evaluación
\newline
Postcondiciones: Se realizan las preasignaciones para un ejercicio de evaluación
\newline
Escenario principal:
\begin{itemize}
	\item 1. El sistema muestra una lista desplegable con los ejercicio de evaluación para los cuales aun no se han realizado preasignaciones existentes en el sistema.
	\item 2. El usuario selecciona uno de los ejercicios de evaluación
	\item 3. El usuario hace clic en el botón de hacer preasignaciones
	\item 4. El sistema realiza el procesado de ráfagas y posteriormente asigna las ediciones del ejercicio de evaluación a los Estudiantes según la configuración del sistema.
	\item 5. El sistema actualiza la pagina de administración de ejercicios de evaluación, mostrando los cambios efectuados.
\end{itemize}
Escenario alternativo 1: 
\begin{itemize}
	\item *.* En cualquier momento del caso de uso el usuario usa los botones de navegación (atrás, adelante, actualizar) y termina el proceso actual sin guardar los datos ni ejecutarse ninguna acción.
\end{itemize}
Escenario alternativo 2:
\begin{itemize}
	\item *.* En cualquier momento del caso de uso el usuario viaja a otra sección de la web o a otra web y termina el proceso actual sin guardar los datos ni ejecutarse ninguna acción.
\end{itemize}

Caso de uso: \textbf{Procesar ráfagas}
\newline
Descripción: Se procesan las posibles ráfagas existentes en el sistema
\newline
Actores: Sistema
\newline
Precondiciones: El usuario selecciona la opción de hacer preasignaciones.
\newline
Postcondiciones: Se analizan las ediciones existentes y se crean las ráfagas que procedan.
\newline
Escenario principal:
\begin{itemize}
	\item 1. El sistema busca en cada pagina todas las ediciones realizadas en el plazo establecido.
	\item 2. El sistema detecta que 2 ediciones consecutivas han sido realizadas en menos de el periodo de tiempo establecido por el mismo usuario.
	\item 3. El sistema crea una ráfaga o añade la ultima edición a una ráfaga existente.
\end{itemize}

\newpage

\section{Modelo de Comportamiento}

En las siguientes imágenes podemos ver los modelos de comportamiento:
\newline

\begin{figure}[h!]
	\centering
	\includegraphics[width=0.6\textwidth]{ds_administrar_roles.jpg}
	\caption{Caso de uso: administrar roles.}
\end{figure}

\begin{figure}
	\centering
	\includegraphics[width=0.6\textwidth]{ds_crear_rol.jpg}
	\caption{Caso de uso: crear rol.}
\end{figure}

\begin{figure}
	\centering
	\includegraphics[width=0.6\textwidth]{ds_modificar_rol.jpg}
	\caption{Caso de uso: editar permisos de rol.}
\end{figure}

\begin{figure}
	\centering
	\includegraphics[width=0.6\textwidth]{ds_eliminar_rol.jpg}
	\caption{Caso de uso: eliminar rol.}
\end{figure}

\begin{figure}
	\centering
	\includegraphics[width=0.6\textwidth]{ds_administrar_roles.jpg}
	\caption{Caso de uso: asignar rol.}
\end{figure}

\begin{figure}
	\centering
	\includegraphics[width=0.6\textwidth]{ds_administrar_ejercicios_evaluacion.jpg}
	\caption{Caso de uso: administrar ejercicios de evaluación.}
\end{figure}

\begin{figure}
	\centering
	\includegraphics[width=0.6\textwidth]{ds_crear_ejercicio_evaluacion.jpg}
	\caption{Caso de uso: crear ejercicio de evaluación.}
\end{figure}

\begin{figure}
	\centering
	\includegraphics[width=0.6\textwidth]{ds_modificar_ejercicio_evaluacion.jpg}
	\caption{Caso de uso: editar ejercicio de evaluación.}
\end{figure}

\begin{figure}
	\centering
	\includegraphics[width=0.6\textwidth]{ds_eliminar_ejercicio_evaluacion.jpg}
	\caption{Caso de uso: eliminar ejercicio de evaluación.}
\end{figure}

\begin{figure}
	\centering
	\includegraphics[width=0.6\textwidth]{ds_editar_criterios_de_evaluacion.jpg}
	\caption{Caso de uso: editar criterios de evaluación.}
\end{figure}

\begin{figure}
	\centering
	\includegraphics[width=0.6\textwidth]{ds_hacer_preasignaciones.jpg}
	\caption{Caso de uso: hacer preasignaciones.}
\end{figure}

\begin{figure}
	\centering
	\includegraphics[width=0.6\textwidth]{ds_procesar_rafagas.jpg}
	\caption{Caso de uso: procesar ráfagas.}
\end{figure}

\begin{figure}
	\centering
	\includegraphics[width=0.6\textwidth]{ds_acceso_ayuda.jpg}
	\caption{Caso de uso: acceso a ayuda.}
\end{figure}

\newpage

\section{Modelo de Interfaz de Usuario}

En esta sección se adjuntan capturas de pantalla de la interfaz del sistema:
\newline

\begin{figure}[h!]
	\centering
	\includegraphics[width=0.9\textwidth]{sc_login.jpg}
	\caption{Pantalla de login.}
\end{figure}

\begin{figure}
	\centering
	\includegraphics[width=0.9\textwidth]{sc_revision.jpg}
	\caption{Pantalla de inicio del usuario (sin la sala de debug, eso es debido a que el modo de desarrollo esta activado).}
\end{figure}

\begin{figure}
	\centering
	\includegraphics[width=0.9\textwidth]{sc_no_revision.jpg}
	\caption{Pantalla de usuario sin revisiones.}
\end{figure}

\begin{figure}
	\centering
	\includegraphics[width=0.9\textwidth]{sc_metaevaluar.jpg}
	\caption{Pantalla de metaevaluaciones.}
\end{figure}

\begin{figure}
	\centering
	\includegraphics[width=0.9\textwidth]{sc_metaevaluar_lista.jpg}
	\caption{Pantalla de listado de metaevaluaciones.}
\end{figure}

\begin{figure}
	\centering
	\includegraphics[width=0.9\textwidth]{sc_student_list.jpg}
	\caption{Pantalla de listado de alumnos.}
\end{figure}

\begin{figure}
	\centering
	\includegraphics[width=0.9\textwidth]{sc_parametros1.jpg}
	\includegraphics[width=0.9\textwidth]{sc_parametros2.jpg}
	\caption{Pantalla de parámetros.}
\end{figure}

\begin{figure}
	\centering
	\includegraphics[width=0.9\textwidth]{sc_roles1.jpg}
	\includegraphics[width=0.9\textwidth]{sc_roles2.jpg}
	\caption{Pantalla de configuración de roles.}
\end{figure}

\begin{figure}
	\centering
	\includegraphics[width=0.9\textwidth]{sc_ee1.jpg}
	\includegraphics[width=0.9\textwidth]{sc_ee2.jpg}
	\caption{Pantalla de configuración de ejercicios de evaluación.}
\end{figure}

\begin{figure}
	\centering
	\includegraphics[width=0.9\textwidth]{sc_extra_help.jpg}
	\caption{Pantalla de ayuda extra.}
\end{figure}

\chapter{Diseño del Sistema}
% ------------------------------------------------------------------------------
% Este fichero es parte de la plantilla LaTeX para la realización de Proyectos
% Final de Grado, protegido bajo los términos de la licencia GFDL.
% Para más información, la licencia completa viene incluida en el
% fichero fdl-1.3.tex

% Copyright (C) 2012 SPI-FM. Universidad de Cádiz
% ------------------------------------------------------------------------------

\section{Arquitectura del Sistema}
La estructura del sistema esta basada en el esquema de modelo vista controlador, introducidos en \href{http://www.codeigniter.com/}{CodeIgniter} ademas de una base de datos propia y la base de datos del media wiki, de forma que encontraremos los archivos principales en las carpetas de vista, donde se genera la interfaz para el usuario, modelo, donde se encuentran las funciones y algoritmo para el funcionamiento interno, y controladores, donde hacemos que se comuniquen las vistas con los modelos.

\subsection{Arquitectura Física}
Son necesarios para nuestro sistema un ordenador que actué de servidor (puede valer donde esté montado el Media Wiki) y el ordenador propio de los usuarios, con los cuales interactuaran con el sistema.

En principio AssessMediaWiki esta pensado para ser independiente del sistema operativo, decisión que verse afectada en un futuro dependiendo de las necesidades de escalabilidad y las funciones que se quieran implementar.

\subsection{Arquitectura Lógica}
La arquitectura del sistema se divide en las siguientes capas

\paragraph*{Modelo}
En este grupo podemos encontrar todo el software que genera las funciones para obtener, generar datos e interactuar con las bases de datos.

\paragraph*{Vista}
En este grupo podemos encontrar todo el software que genera la interfaz del usuario, siendo esta su funcionalidad exclusiva.

\paragraph*{Controlador}
Aquí encontraremos el software que completa los campos de las vistas, comunicándolas con los modelos. Los controladores se encargan de mostrar los datos del sistema en las vistas, e interaccionar con los usuarios, ya sea modificando dichos datos o navegando entre las distintas vistas.

\paragraph*{Bases de datos}
Nuestro sistema cuenta con dos bases de datos fundamentales, la base de datos del MediaWiki, donde se guarda toda la información de los alumnos y su aportación al Wiki, y la base de datos del AssessMediaWiki, donde almacenaremos la nueva información generada por la interacción de los usuarios.

\begin{figure}[h!]
	\centering
	\includegraphics[width=0.9\textwidth]{db2.jpg}
	\caption{Diagrama de la base de datos de AMW 2.0.}
\end{figure}

\begin{figure}[h!]
	\centering
	\includegraphics[width=0.9\textwidth]{mediawikidb.png}
	\caption{Diagrama de la base de datos de MediaWiki (disponible en la web de MediaWiki).}
\end{figure}

\newpage

\section{Parametrización del software base}
Sera necesario editar el archivo de configuración para introducir los datos de nuestro MediaWiki, permitiendo así acceso a la base de datos y a generar las URLs para las evaluaciones (este proceso se explica detalladamente en el manual de instalación). También tenemos la opción de activar o desactivar el modo de desarrollo para hacer pruebas o cambios según se requiera.

\section{Diseño Físico de Datos}
En esta sección se define la estructura física de datos que utilizará el sistema, a partir del modelo de conceptual de clases, de manera que teniendo presente los requisitos establecidos para el sistema de información y las particularidades del entorno tecnológico, se consiga un acceso eficiente de los datos.
La estructura física se compone de tablas, índices, procedimientos almacenados, secuencias y otros elementos dependientes del SGBD a utilizar.

\begin{figure}[!h]
	\centering
	\includegraphics[width=0.9\textwidth]{db2.jpg}
	\caption{Diagrama de la base de datos de AMW 2.0.}
\end{figure}




\chapter{Construcción del Sistema}
% ------------------------------------------------------------------------------
% Este fichero es parte de la plantilla LaTeX para la realización de Proyectos
% Final de Grado, protegido bajo los términos de la licencia GFDL.
% Para más información, la licencia completa viene incluida en el
% fichero fdl-1.3.tex

% Copyright (C) 2012 SPI-FM. Universidad de Cádiz
% ------------------------------------------------------------------------------

Este capítulo trata sobre todos los aspectos relacionados con la implementación del sistema en código, haciendo uso de un determinado entorno tecnológico.

\section{Entorno de Construcción}
En esta sección se debe indicar el marco tecnológico utilizado para la construcción del sistema: entorno de desarrollo (IDE), lenguaje de programación, herramientas de ayuda a la construcción y despliegue, control de versiones, repositorio de componentes, integración contínua, etc.

Para desarrollar este proyecto hemos usado:

\begin{itemize}
	\item \href{http://www.codeigniter.com/}{CodeIgniter} [\url{http://www.codeigniter.com/}]
	\item \href{https://secure.php.net/}{PHP} [\url{https://secure.php.net/}]
	\item HTML[\url{https://es.wikipedia.org/wiki/HTML}]
	\item \href{https://www.mysql.com/}{MYSQL} [\url{https://www.mysql.com/}]
	\item \href{https://forja.rediris.es/}{Forja Rediris} [\url{https://forja.rediris.es/}]
	\item \href{http://tortoisesvn.net/}{Tortoise SVN} [\url{http://tortoisesvn.net//}]
	\item \href{https://www.mozilla.org/es-ES/}{Firefox} [\url{https://www.mozilla.org/es-ES/}]
	\item \href{http://www.latex-project.org/}{LaTeX} [\url{http://www.latex-project.org/}]
	\item \href{http://www.texstudio.org/}{TeXtudio} [\url{http://www.texstudio.org/}]
	\item \href{http://dia-installer.de/index.html.es}{Dia} [\url{http://dia-installer.de/index.html.es}]	
\end{itemize}

\section{Código Fuente}
Organización del código fuente, describiendo la utilidad de los diferentes ficheros y su distribución en paquetes o directorios. Asimismo, se incluirá algún extracto significativo de código fuente que sea de interés para ilustrar algún algoritmo o funcionalidad específica del sistema.\\

Se ha usado una estructura modelo vista controlador, de forma que al analizar inicialmente el sistema, pude ver que en la carpeta de vista guardaremos las interfaces, en la de modelos guardaremos los objetos donde se definen todas las funciones y en la carpeta de controladores guardaremos los controladores, que hacen posible que vistas y modelos interactúen entre si.\\

También me gustaría destacar el algoritmo de barajado aleatorio usado para la preasignación de ediciones a los alumnos.\\
Fuente: 
\href{https://es.wikipedia.org/wiki/Algoritmo_Fisher-Yates}{Wikipedia: Algoritmo Fisher-Yates}

Otra cosa a destacar es que con el antiguo esquema de la base de datos, al realizarse una re-evaluación, esta no era tratada de forma especial, y contaba como si fuese una evaluación normal de una edición.
Gracias a la nueva distribución de la base de datos se ha solventado ese problema, manteniendo a su vez la compatibilidad con otros sistemas como StatMediaWiki o CleverFigures

\section{Scripts de Base de datos}
El script para la creación de la base de datos, así como para introducir los datos y valores iniciales es el siguiente:\\

-- --------------------------------------------------------\\

--\\
-- Estructura de tabla para la tabla `config`\\
--\\

CREATE TABLE IF NOT EXISTS `config` (\\
`parameter` varchar(50) NOT NULL,\\
`value` varchar(50) NOT NULL,\\
PRIMARY KEY (`parameter`)\\
) ENGINE=InnoDB DEFAULT CHARSET=utf8;\\

-- --------------------------------------------------------\\

--\\
-- Estructura de tabla para la tabla `entregables`\\
--\\

CREATE TABLE IF NOT EXISTS `entregables` (\\
`ent\_id` int(11) NOT NULL AUTO\_INCREMENT,\\
`ent\_entregable` varchar(250) NOT NULL,\\
`ent\_description` varchar(255) NOT NULL,\\
`generic\_specific` boolean NOT NULL DEFAULT false, --0 = generic, 1 = specific\\
PRIMARY KEY (`ent\_id`)\\
) ENGINE=InnoDB  DEFAULT CHARSET=utf8 AUTO\_INCREMENT=1 ;\\

-- --------------------------------------------------------\\

--\\
-- Estructura de tabla para la tabla `evaluaciones`\\
--\\

CREATE TABLE IF NOT EXISTS `evaluaciones` (\\
`eva\_id` int(11) NOT NULL AUTO\_INCREMENT,\\
`eva\_user` int(11) NOT NULL,\\
`eva\_revisor` int(11) NOT NULL,\\
`eva\_revision` int(11) NOT NULL,\\
`eva\_time` int(11) NOT NULL,\\
PRIMARY KEY (`eva\_id`)\\
) ENGINE=InnoDB  DEFAULT CHARSET=utf8 AUTO\_INCREMENT=1 ;\\

-- --------------------------------------------------------\\

--\\
-- Estructura de tabla para la tabla `evaluaciones\_entregables`\\
--\\

CREATE TABLE IF NOT EXISTS `evaluaciones\_entregables` (\\
`eva\_id` int(11) NOT NULL,\\
`ent\_id` int(11) NOT NULL,\\
`ee\_nota` int(11) NOT NULL,\\
`ee\_comentario` varchar(250) NOT NULL,\\
PRIMARY KEY (`eva\_id`,`ent\_id`)\\
) ENGINE=InnoDB DEFAULT CHARSET=utf8;\\

-- --------------------------------------------------------\\

--\\
-- Estructura de tabla para la tabla `replies`\\
--\\

CREATE TABLE IF NOT EXISTS `replies` (\\
`rep\_id` int(11) NOT NULL AUTO\_INCREMENT,\\
`rep\_read` int(11) NOT NULL,\\
`rep\_new` int(11) NOT NULL,\\
PRIMARY KEY (`rep\_id`)\\
) ENGINE=InnoDB  DEFAULT CHARSET=utf8 AUTO\_INCREMENT=1 ;\\

-- --------------------------------------------------------\\

--\\
-- Estructura de tabla para la tabla `metaevaluaciones`\\
--\\

CREATE TABLE IF NOT EXISTS `metaevaluaciones` (\\
`mev\_id` int(11) NOT NULL AUTO\_INCREMENT,\\
`mevaluador\_id` int(11) NOT NULL,\\
`evaluacion\_id` int(11) NOT NULL,\\
`calificacion` int(11) NOT NULL,\\
`comentario` varchar(250) NOT NULL,\\
PRIMARY KEY (`mev\_id`)\\
) ENGINE=InnoDB  DEFAULT CHARSET=utf8 AUTO\_INCREMENT=1 ;\\

-- --------------------------------------------------------\\

--\\
-- Estructura de tabla para la tabla `roles`\\
--\\

CREATE TABLE IF NOT EXISTS `roles` (\\
`rol\_id` int(11) NOT NULL AUTO\_INCREMENT,\\
`name` varchar(30) NOT NULL UNIQUE,\\
`evaluar` boolean NOT NULL DEFAULT true,\\
`feedback` boolean NOT NULL DEFAULT true,\\
`metaevaluar` boolean NOT NULL DEFAULT false,\\
`metaevaluar\_lista` boolean NOT NULL DEFAULT false,\\
`alumnos` boolean NOT NULL DEFAULT false,\\
`parametros` boolean NOT NULL DEFAULT false,\\
PRIMARY KEY (`rol\_id`)\\
) ENGINE=InnoDB  DEFAULT CHARSET=utf8 AUTO\_INCREMENT=1 ;\\

-- --------------------------------------------------------\\

--\\
-- Estructura de tabla para la tabla `rol\_assignation`\\
--\\

CREATE TABLE IF NOT EXISTS `rol\_assignation` (\\
`user\_id` int(11) NOT NULL,\\
`rol\_id` int(11) NOT NULL,\\
PRIMARY KEY (`user\_id`)\\
) ENGINE=InnoDB DEFAULT CHARSET=utf8 AUTO\_INCREMENT=1 ;\\

-- --------------------------------------------------------\\

--\\
-- Estructura de tabla para la tabla `ejercicios\_de\_evaluacion`\\
--\\

CREATE TABLE IF NOT EXISTS `ejercicios\_de\_evaluacion` (\\
`evaluation\_id` int(11) NOT NULL AUTO\_INCREMENT,\\
`exercise\_name` varchar(30) NOT NULL UNIQUE,\\
`beginning` date NOT NULL,\\
`first\_phase\_end` date NOT NULL,\\
`second\_phase\_end` date NOT NULL,\\
`third\_phase\_end` date NOT NULL,\\
`fourth\_phase\_end` date NOT NULL,\\
`description` varchar(500) NOT NULL,\\
PRIMARY KEY (`evaluation\_id`)\\
) ENGINE=InnoDB DEFAULT CHARSET=utf8 AUTO\_INCREMENT=1 ;\\

-- ----------------------------------------------------------\\

--\\
-- Estructura de tabla para la tabla `preasignaciones`\\
--\\

CREATE TABLE IF NOT EXISTS `preasignaciones` (\\
`preasignacion\_id` int(11) NOT NULL AUTO\_INCREMENT,\\
`edit\_id` int(11) NOT NULL,\\
`revisor\_id` int(11) NOT NULL,\\
`ejercicio\_de\_evaluacion` int(11) NOT NULL,\\
PRIMARY KEY (`preasignacion\_id`)\\
) ENGINE=InnoDB DEFAULT CHARSET=utf8 AUTO\_INCREMENT=1 ;\\

-- --------------------------------------------------------\\

--\\
-- Estructura de tabla para la tabla `categorias\_ej\_ev`\\
--\\

CREATE TABLE IF NOT EXISTS `categorias\_ej\_ev` (\\
`evaluation\_id` int(11) NOT NULL,\\
`ent\_id` int(11) NOT NULL,\\
PRIMARY KEY (`evaluation\_id`,`ent\_id`)\\
) ENGINE=InnoDB DEFAULT CHARSET=utf8 AUTO\_INCREMENT=1 ;\\

-- --------------------------------------------------------\\

--\\
-- Estructura de tabla para la tabla `rafagas`\\
--\\

CREATE TABLE IF NOT EXISTS `rafagas` (\\
`raf\_start` int(8) NOT NULL,\\
`raf\_end` int(8) NOT NULL,\\
`raf\_timestamp` char(14) NOT NULL,\\
`raf\_size` int(10) NOT NULL,\\
PRIMARY KEY (`raf\_start`)\\
) ENGINE=InnoDB DEFAULT CHARSET=utf8;\\

-- --------------------------------------------------------\\


--\\
-- Insercion de valores por defecto\\
--
-- A la vez que creamos la tabla añadimos los dos primeros usuarios, esto ha de hacerse como ultima accion\\
-- ya que si esta creado dara error y no se crearan las siguientes tablas, y una vez añadidos los usuarios\\
-- añadimos al primer usuario creado en la wiki como administrador.\\
-- \#TODO evitar esse error

INSERT INTO `roles`(`name`, `evaluar`, `feedback`, `metaevaluar`, `metaevaluar\_lista`, `alumnos`,\\ `parametros`)\\
VALUES ("Admin",1,1,1,1,1,1);\\
INSERT INTO `roles`(`name`, `evaluar`, `feedback`)  \\
VALUES ("Student",1,1);\\
INSERT INTO `rol\_assignation`(`user\_id`, `rol\_id`)\\
VALUES (1,1);\\

-- --------------------------------------------------------\\


La mayoría de funciones para manejar las bases de datos se encuentran en los modelos del sistema.\\

Como se ha mencionado anteriormente, al modificar la estructura de la base de datos del sistema (ver [Fig.6.1] y [Fig.6.2] a continuación) se ha conseguido solventar algunas debilidades de la versión anterior, así como implementar mejoras como las ráfagas y las preasignaciones, para las cuales es necesario que las bases de datos de AssessMediaWiki y MediaWiki se intercomuniquen, ese proceso se puede observar con mas detenimiento en el modelo de ejercicios de evaluación.

\clearpage

\begin{figure}
	\centering
	\includegraphics[width=0.6\textwidth]{db1girada.jpg}
	\caption{Diagrama de la base de datos de AMW 1.0.}
\end{figure}

\begin{figure}
	\centering
	\includegraphics[width=0.6\textwidth]{db2girada.jpg}
	\caption{Diagrama de la base de datos de AMW 2.0.}
\end{figure}

\clearpage

\chapter{Pruebas del Sistema}
% ------------------------------------------------------------------------------
% Este fichero es parte de la plantilla LaTeX para la realización de Proyectos
% Final de Grado, protegido bajo los términos de la licencia GFDL.
% Para más información, la licencia completa viene incluida en el
% fichero fdl-1.3.tex

% Copyright (C) 2012 SPI-FM. Universidad de Cádiz
% ------------------------------------------------------------------------------

\section{Estrategia}
Para realizar las pruebas de sistema se ha creado un MediaWiki local con varios usuarios y ediciones.

\section{Entorno de Pruebas}
Para realizar pruebas en el sistema es necesario al menos un MediaWiki con usuarios y ediciones y que se haya creado la propia base de datos del sistema.

\section{Roles}
Describir en esa sección cuáles serán los perfiles y participantes necesarios para la ejecución de cada uno de los niveles de prueba.\\

Los roles de pruebas básicos son el de Estudiante y Profesor, pero dado que se pueden crear roles personalizables en el sistema es imposible hacer un listado total de los roles.

\section{Niveles de Pruebas}
En este sección se documentan los diferentes tipos de pruebas que se han llevado a cabo, ya sean manuales o automatizadas mediante algún software específico de pruebas.\\

Todas las pruebas de inserción de datos han sido realizadas manualmente, así como la corrección de ediciones y la creación y edición tanto de roles como de ejercicios de evaluación

\subsection{Pruebas de Sistema}
Se creo un apartado: "sala de depurado (debug room)" para realizar pruebas preliminares introduciendo datos manualmente en el código en vez de hacerlas con la interfaz, lo cual agilizo mucho el proceso de realización de pruebas, permitiendo hacer varias pruebas de forma simultanea en el sistema.

\subsection{Pruebas de Aceptación}
El objetivo de estas pruebas es demostrar que el producto está listo para el paso a producción. Suelen ser las mismas pruebas que se realizaron anteriormente pero en el entorno de producción. En estas pruebas, es importante la participación del cliente final.\\

Al realizar las pruebas con la interfaz se vieron los mismos errores que en la sala de depurado, principalmente de comunicación con las bases de datos, una vez solventados estos problemas en la sala de depurado se comprobó que también hubiesen sido solventados en la interfaz del usuario.

% EPILOGO
\part{Epílogo}
\null\vfill
\noindent En esta última parte quedarán recogidas las conclusiones y los manuales necesarios para el manejo de la aplicación resultado del desarrollo. Si se ha realizado algún tipo de evaluación de la solución proporcionada, más allá de las pruebas del sistema, también deberá venir recogida en un capítulo separado dentro de esta parte. Pueden consultarse diversos tipos de evaluaciones sobre sistemas de información en \cite{hevner2004}: casos de estudio, análisis estático, análisis dinámico, simulación, experimento controlado, etc.
\vfill

\chapter{Manual de implantación y explotación}
% ------------------------------------------------------------------------------
% Este fichero es parte de la plantilla LaTeX para la realización de Proyectos
% Final de Grado, protegido bajo los términos de la licencia GFDL.
% Para más información, la licencia completa viene incluida en el
% fichero fdl-1.3.tex

% Copyright (C) 2012 SPI-FM. Universidad de Cádiz
% ------------------------------------------------------------------------------

Las instrucciones de instalación y explotación del sistema se detallan a continuación.

\section{Introducción}
AssessMediaWiki es una aplicación web de código abierto que, al conectarse a una instalación MediaWiki, proporciona procedimientos de autoevaluación, hetero evaluación y evaluación entre iguales, a la vez que mantiene información sobre esas evaluaciones. Los supervisores pueden obtener informes que ayudan en la evaluación de los estudiantes.
\newline

Aunque hay un gran número de extensiones para el sistema MediaWiki, no hemos encontrado ninguna que permitiera evaluar contribuciones individuales a un wiki. La mayoría de las aproximaciones solo ofrecen formas de evaluar una versión en particular de un artículo (normalmente la más reciente), siendo ineficaces en este caso. Por ello, para evaluar la calidad de las contribuciones creamos AssessMediaWiki.
\newline

AssessMediaWiki implementa como base dos roles de usuario distintos: supervisores y estudiantes. Los estudiantes pueden elegir entre distintas opciones: evaluar una revisión, comprobar sus propias aportaciones evaluadas y verificar las evaluaciones ya enviadas. Por otro lado, los supervisores tienen un mayor número de opciones, como modificar los parámetros de los programas o vigilar las evaluaciones que los alumnos vayan haciendo.
\newline

\section{Requisitos previos}
Para usar AssessMediaWiki es necesaria la previa presencia de un MediaWiki donde los alumnos vayan a trabajar

\section{Inventario de componentes}
Lista de los componentes hardware y software que se incluyen en la versión del producto:
\begin{itemize}
	\item AssessMediaWiki 2.0
	\item Manual de instalación.
	\item Ayuda al usuario.
\end{itemize}

\section{Procedimientos de instalación}
\textbf{Descarga e instalación}

AssessMediaWiki se puede descargar desde su \href{https://forja.rediris.es/projects/assessmediawiki/}{web oficial}, dentro de la pestaña de \href{http://forja.rediris.es/frs/?group_id=1135/}{descargas}. El contenido del archivo se debe descomprimir en la carpeta que desee del servidor web (que permita ejecutar ficheros de lenguaje PHP). Por ejemplo, en el caso de los paquetes Xampp se denomina htdocs, y en otras instalaciones de Apache www.\\


\textbf{Configuración previa}\\

Tras instalar AssesMediaWiki en nuestro equipo tendremos que entrar en los siguientes ficheros para hacer las siguientes modificaciones para que funcione adecuadamente:\\

htcdocs/assesmediawiki/applications/config/amw.php:\\

$config["database_mw"] = "mediawikidb";$ mediawikidb sería la base de datos propia del mismo wiki\\

$config["username_mw"] = "user";$ user sera el nombre de usuario de mysql\\

$config["password_mw"] = "password";$ password sera la contraseña de mysql\\

$config["usuarios_admin"] = array(1, 2);$ en la array indicamos cuales usuarios de la wiki serán los administradores del AssesMediaWiki, en este caso serian el primero y el segundo en registrarse, los cuales saldrán por ese orden en la base de datos del wiki\\


htcdocs/assesmediawiki/applications/config/database.php:\\

db['default']['hostname'] = 'localhost'; localhost será el servidor donde se encuentran ubicadas las bases de datos, en caso de estar en el propio equipo lo dejaremos tal cual, en caso contrario lo cambiaremos por la dirección del servidor.\\

$config["username_mw"] = "user";$ user sera el nombre de usuario de mysql\\

$config["password_mw"] = "password";$ password sera la contraseña de mysql\\

$db['default']['database'] = 'amw';$ amw sería la base de datos propia de AssesMediaWiki, generada por el mismo programa. (En caso de que el AssesMediaWiki no cree su propia base de datos podemos generarla con la sentencia $“mysql nombre_base_de_datos < fichero_de_texto”$, pudiendo encontrar el fichero de texto en:$ htcdocs/assesmediawiki/application/sql/estructura_amw$ y siendo en este caso $“amw”$ el nombre de la base de datos).\\

\textbf{Configuración}\\

Tras haber configurado dichos archivos ejecutaremos el sistema gestor de bases de datos y el servidor web (en este orden). Por ejemplo, en el panel del control de Xampp o usando los scripts de /etc/init.d (o los comandos start, stop, reload y restart que lo sustituyen en “upstart”).\\

Tras hacer entrar al sistema como usuario administrador iremos a la pestaña de “Parameters”, donde modificaremos los siguientes datos:\\

Start date: fecha de inicio de evaluación de las entradas en la wiki.\\

End date: fecha de fin de evaluación de las entradas en la wiki.\\

Evals per student: numero de entradas que evaluara cada estudiante.\\

Meta-vals per student: numero de evaluaciones que evaluara cada metaevaluador.\\

Wiki URL: dirección URL del wiki en el que usaremos el AssesMediaWiki, es muy importante poner bien la dirección URL, ya que sin esta el AssesMediaWiki no puede funcionar correctamente.\\

\section{Pruebas de implantación}
Seria recomendable probar a crear y editar algún rol y ejercicio de evaluación y dentro del MediaWiki algún usuario y una edición de prueba para intentar corregirla.

\section{Procedimientos de operación y nivel de servicio}
Procedimientos necesarios para asegurar el correcto funcionamiento, rendimiento, disponibilidad y seguridad del sistema: back-ups, chequeo de logs, etc. También, es preciso indicar claramente aquellas actuaciones precisas necesarias para el mantenimiento preventivo del sistema y así prevenir posibles fallos en el mismo.\\

Para asegurar el correcto funcionamiento sera necesario tener el servidor siempre encendido y con conexión a internet, para que así sea posible acceder al sistema en cualquier momento y desde cualquier lugar.\\
Asimismo se recomienda la creación de una copia de seguridad periódicamente, para prevenir la perdida de datos ante algún error desconocido o problema con el servidor.


\chapter{Manual de usuario}
% ------------------------------------------------------------------------------
% Este fichero es parte de la plantilla LaTeX para la realización de Proyectos
% Final de Grado, protegido bajo los términos de la licencia GFDL.
% Para más información, la licencia completa viene incluida en el
% fichero fdl-1.3.tex

% Copyright (C) 2012 SPI-FM. Universidad de Cádiz
% ------------------------------------------------------------------------------

\section{Introducción}
AssessMediaWiki es una aplicación web de código abierto que, al conectarse a una instalación MediaWiki, proporciona procedimientos de autoevaluación, hetero evaluación y evaluación entre iguales, a la vez que mantiene información sobre esas evaluaciones. Los supervisores pueden obtener informes que ayudan en la evaluación de los estudiantes.
\newline

Aunque hay un gran número de extensiones para el sistema MediaWiki, no hemos encontrado ninguna que permitiera evaluar contribuciones individuales a un wiki. La mayoría de las aproximaciones solo ofrecen formas de evaluar una versión en particular de un artículo (normalmente la más reciente), siendo ineficaces en este caso. Por ello, para evaluar la calidad de las contribuciones creamos AssessMediaWiki.
\newline

AssessMediaWiki implementa como base dos roles de usuario distintos: supervisores y estudiantes. Los estudiantes pueden elegir entre distintas opciones: evaluar una revisión, comprobar sus propias aportaciones evaluadas y verificar las evaluaciones ya enviadas. Por otro lado, los supervisores tienen un mayor número de opciones, como modificar los parámetros de los programas o vigilar las evaluaciones que los alumnos vayan haciendo.
\newline

\section{Características}
Las principales funcionalidades del sistema son:
\begin{itemize}
	\item Configurar ejercicios de evaluación.
	\item Evaluar ediciones al azar (dentro de las mas significativas).
	\item Generar CSV.
\end{itemize}

\section{Requisitos previos}
EL requisito principal es que debe haber un MediaWiki para que los alumnos trabajen sobre el.

\section{Uso del sistema}
Lo único necesario para poder usar el sistema es estar logeado en el MediaWiki existente, tras eso tan solo hay que seguir el manual de instalación y una vez se haya configurado el sistema podemos proceder a realizar pruebas del mismo, creando usuarios en el MediaWiki y realizando ediciones para posteriormente evaluarlas.\\

Lo primero que nos encontramos al entrar en AssessMediaWiki es la pantalla de login, aquí los usuarios se registraran con las mismas cuentas que usan en el MediaWiki.\\

\begin{figure}[h!]
	\centering
	\includegraphics[width=0.9\textwidth]{sc_login.jpg}
	\caption{Pantalla de login.}
\end{figure}
\clearpage

Una vez iniciada sesión, si hay algún ejercicio de evaluación con preasignaciones realizadas, se mostrara a los alumnos un texto con un enlace a la edición que tienen que evaluar y varios campos con la información que deben rellenar.\\

\begin{figure}[h!]
	\centering
	\includegraphics[width=0.9\textwidth]{sc_revision.jpg}
	\caption{Pantalla de inicio del usuario (sin la sala de debug, eso es debido a que el modo de desarrollo esta activado).}
\end{figure}
\clearpage

Por el contrario, si no tienen ninguna edición que evaluar se mostrara el siguiente mensaje:\\

\begin{figure}[h!]
	\centering
	\includegraphics[width=0.9\textwidth]{sc_no_revision.jpg}
	\caption{Pantalla de usuario sin revisiones.}
\end{figure}
\clearpage

Si así lo deseamos podemos modificar los permisos del rol de alumnos o crear un rol especial para que algunos alumnos realicen metaevaluaciones sobre las evaluaciones realizadas.\\

su interfaz es similar a los anteriores, solo que en este caso no deberán evaluar varios criterios de la edición, sino que deberán evaluar en general la evaluación realizada sobre la edición\\

\begin{figure}[h!]
	\centering
	\includegraphics[width=0.9\textwidth]{sc_metaevaluar.jpg}
	\caption{Pantalla de metaevaluaciones.}
\end{figure}
\clearpage

Dependiendo del grado de responsabilidad y confianza que queramos darle al alumno podemos darle acceso a la lista de metaevaluaciones, si no esta lista sera solo accesible para el profesor / administrador, en la cual podrá ver todas las metaevaluaciones que han sido realizadas.\\

\begin{figure}[h!]
	\centering
	\includegraphics[width=0.9\textwidth]{sc_metaevaluar_lista.jpg}
	\caption{Pantalla de listado de metaevaluaciones.}
\end{figure}
\clearpage

En el apartado de la lista de estudiantes podemos ver el resumen de actividad de cada estudiante, ya sea por evaluaciones o metaevaluaciones realizadas y recibidas. También podemos generar un CSV (Coma Separated Values) con la información del alumno.\\

\begin{figure}[h!]
	\centering
	\includegraphics[width=0.9\textwidth]{sc_student_list.jpg}
	\caption{Pantalla de listado de alumnos.}
\end{figure}
\clearpage

A continuación tenemos los parámetros de AssessMediaWiki, aquí podemos definir las categorías a evaluar, si son genéricas o especificas y una lista de parámetros entre los que encontraremos los siguientes en esta nueva versión:\\

\begin{itemize}
	\item Ediciones evaluadas por alumno - numero de ediciones que serán evaluadas de cada alumno.
	\item Evaluaciones por edición - numero de evaluaciones que recibirá cada edición.
	\item Autoevaluaciones - para permitir que un alumno evalúe su propia edición si se diera el caso.
\end{itemize}

Para realizar cualquier cambio en alguno de estos parámetros solo tenemos que modificarlos y hacer clic en el botón "Modify parameters".\\

A través de este apartado de parámetros también podemos acceder a la administración de roles y ejercicios de evaluación, las cuales veremos a continuación.

\begin{figure}[h!]
	\centering
	\includegraphics[width=0.9\textwidth]{sc_parametros1.jpg}
	\includegraphics[width=0.9\textwidth]{sc_parametros2.jpg}
	\caption{Pantalla de parámetros.}
\end{figure}
\clearpage

En la pagina de administración de roles podremos editar los permisos de los roles existentes, crear roles, eliminarlos, asignarlos a un alumno y ver la lista de asignaciones realizadas.\\

Para modificar los permisos de un rol basta con seleccionar las casillas deseadas y hacer clic en el botón "Edit rol permissions" ubicado al final de la fila. \\

Para añadir un rol solo tenemos que escribir el nombre deseado del rol, seleccionar los permisos que le vamos a dar y hacer clic en el botón "Create rol".\\

Para eliminar un rol tan solo hay que seleccionarlo de la lista y hacer clic en "Delete rol", el propio sistema se encargaría de eliminar todas las asignaciones existentes si las hubiera.\\

Para asignarle un rol a un usuario tan solo debemos seleccionar el rol deseado y el usuario que recibirá dicho rol de las listas desplegables y presionar el botón "Assign rol". Cabe destacar que en la lista posterior a esta opción solo se mostraran aquellos usuarios con roles especiales, es decir, los que no sean estudiantes.\\

\begin{figure}[h!]
	\centering
	\includegraphics[width=0.9\textwidth]{sc_roles1.jpg}
	\includegraphics[width=0.9\textwidth]{sc_roles2.jpg}
	\caption{Pantalla de configuración de roles.}
\end{figure}
\clearpage

En la pagina de administración de ejercicios de evaluación podremos editar las propiedades de los ejercicios de evaluación existentes, crear ejercicios de evaluación, eliminarlos, editar las categorías a evaluar en cada uno de ellos y realizar y eliminar preasignaciones.\\

Para modificar los parámetros de un ejercicio de evaluación basta con modificar los datos de la fila correspondiente y hacer clic en el botón "Edit evaluación exercise" ubicado al final de la fila. \\

Para añadir un ejercicio de evaluación solo tenemos que escribir el nombre deseado del ejercicio de evaluación, introducir los datos deseados y hacer clic en el botón "Create new evaluation exercise".\\

Para eliminar un ejercicio de evaluación tan solo hay que seleccionarlo de la lista y hacer clic en "Delete evaluation exercise", pero a diferencia del caso de los roles el sistema mantendrá todas las evaluaciones referentes al ejercicio de evaluación, lo único que evitara es que modifiquemos los parámetros del mismo.\\

Para realizar o eliminar las preasignaciones de un ejercicio de evaluación tan solo debemos seleccionar el ejercicio de evaluación deseado en la lista desplegable y hacer clic en el botón "Make preasignations" o "Delete preasignations".\\ 

Cabe destacar que en las listas solo aparecerán los ejercicios de evaluación en los cuales no se hayan creado todavía las preasignaciones o se hayan realizado todas las evaluaciones para el caso de la creación de preasignaciones, y para el caso de eliminar las preasignaciones solo aparecerán aquellos ejercicios de evaluación para los que quede alguna preasignación pendiente de evaluar.\\

\begin{figure}[h!]
	\centering
	\includegraphics[width=0.9\textwidth]{sc_ee1.jpg}
	\includegraphics[width=0.9\textwidth]{sc_ee2.jpg}
	\caption{Pantalla de configuración de ejercicios de evaluación.}
\end{figure}
\clearpage

En caso de alguna duda o si fuese necesario un recordatorio, no será necesario consultar este manual una vez instalado el sistema, ya que el propio sistema al final la sección de parámetros tiene enlace a una sección de ayuda para el usuario, con información de las opciones configurables del sistema con las que el docente deberá trabajar.

\begin{figure}[h!]
	\centering
	\includegraphics[width=0.9\textwidth]{sc_extra_help.jpg}
	\caption{Pantalla de ayuda extra.}
\end{figure}
\clearpage







\chapter{Conclusiones}
% ------------------------------------------------------------------------------
% Este fichero es parte de la plantilla LaTeX para la realización de Proyectos
% Final de Grado, protegido bajo los términos de la licencia GFDL.
% Para más información, la licencia completa viene incluida en el
% fichero fdl-1.3.tex

% Copyright (C) 2012 SPI-FM. Universidad de Cádiz
% ------------------------------------------------------------------------------

\section{Objetivos alcanzados}
Este apartado debe resumir los objetivos generales y específicos alcanzados, relacionándolos con todo lo descrito en el capítulo de introducción.\\

Se ha podido ampliar las funcionalidades del sistema AssessMediaWiki, añadiendo mas herramientas para los docentes y mas funciones para los alumnos, con la posibilidad de darles responsabilidades personalizadas y haciendo que la experiencia docente sea mas interactiva entre ellos y con el profesor.

\section{Lecciones aprendidas}
A continuación, se detallan las buenas prácticas adquiridas, tanto tecnológicas como procedimentales, así como cualquier otro aspecto de interés.\\

Resumir cuantitativamente el tiempo y esfuerzo dedicados al proyecto a lo largo de su desarrollo que escribir un sencillo 'he trabajado mucho en este proyecto'.
Tras unos nueve meses (a parte de experiencias previas) dedicándole a este proyecto una media de dos horas diarias (o al menos intentándolo) me he dado cuenta de que realmente lanzar una actualización o una nueva versión de algo lleva mucho tiempo y esfuerzo.\\

También he aprendido mucho sobre los framework, buscar recursos por internet y la ayuda de la comunidad en webs como StackOverflow

\section{Trabajo futuro}
En esta sección, se presentan las diversas áreas u oportunidades de mejora detectadas durante el desarrollo del proyecto y que podrán ser abarcadas en futuras versiones del software.\\

Como trabajo futuro quedan pendiente:
\begin{itemize}
	\item Detección de mas wiki-comportamientos, como las ediciones de correcciones ortográficas y desplazamientos de texto.
	\item Realizar preasignaciones se realicen de forma automática (intentando mantener la independencia con el sistema operativo).
	\item Poder trabajar con AssessMediaWiki vía API.
	\item Integración como una extensión de MediaWiki.
	\item Botón de conflicto de interés / rechazar evaluación.
	\item Estudio de la aplicación de técnicas de Learning Analytics a los resultados cualitativos obtenidos. \cite{Conde} 
\end{itemize}

\chapter*{\bibname}
\addcontentsline{toc}{chapter}{\bibname}
%\renewcommand{\bibname}{}

%\input{./bibliografia}

\begingroup
  \def\chapter*#1{}
\renewcommand{\bibname}{}
% Bibliografía con BibTeX
\bibliographystyle{apalike}
\bibliography{bibliografia}

\backmatter

% ------------------------------------------------------------------------------
% Este fichero es parte de la plantilla LaTeX para la realización de Proyectos
% Final de Grado, protegido bajo los términos de la licencia GFDL.
% Para más información, la licencia completa viene incluida en el
% fichero fdl-1.3.tex

% Copyright (C) 2012 SPI-FM. Universidad de Cádiz
% ------------------------------------------------------------------------------


\chapter*{\rlap{Información sobre Licencia}}
\phantomsection  % so hyperref creates bookmarks
\addcontentsline{toc}{chapter}{Información sobre Licencia}
%\label{label_fdl}

 \begin{center}

      \textbf{Información sobre Licencia}


\end{center}

El producto está licenciado bajo GPL 1.3 y a continuación se expone su licencia.\\

\textbf{Licencia pública general GNU}\\

Versión 3, 29 de junio de 2007\\

Copyright (C) 2007 Free Software Foundation, Inc. \url{http://fsf.org/}\\

Se  permite  la  copia  y  distribución  de  copias  literales  de  esta  licencia,  pero  no está permitido modificarla\\

\textbf{Preámbulo}\\

La Licencia Pública General GNU es una licencia libre, de copia permitida (copy-left) para software y otros tipos de trabajos.\\

Las licencias de la mayoría del software y otros trabajos prácticos están diseñadas para quitarle a usted la libertad de compartir y modificar esos trabajos. Por el contrario, la Licencia Pública General GNU pretende garantizarle la libertad para compartir y modificar todas las versiones de un programa y asegurar que permanecerá como software libre para todos sus usuarios. Nosotros, La Fundación Software Libre, usamos la Licencia Pública General GNU para la mayoría de nuestro software; también se aplica a cualquier otro trabajo liberado de esta forma por sus autores. Usted también puede aplicarla a sus programas.\\

Cuando hablamos de software libre, nos referimos a libertad, no al precio. Nuestras Licencias Públicas Generales están diseñadas para asegurar que usted tenga la libertad  para  distribuir  copias  de  software  libre  (y  cobrar  por  ello  si  quiere),  que reciba  el  código  fuente  o  pueda  obtenerlo  si  usted  quiere,  que  pueda  modificar  el
software o usar parte del mismo en nuevos programas libres, y que sepa que pueda hacer estas cosas.\\

Para proteger sus derechos, necesitamos impedir que otros le nieguen estos derechos o le soliciten renunciar a ellos. Por lo tanto, usted tiene ciertas responsabilidades si distribuye copias del software, o si usted lo modifica: la responsabilidad para respetar la libertad de otros.\\

Por ejemplo, si distribuye copias de tales programas, gratuitamente o no, debe transmitir a los destinatarios los mismos derechos que usted recibió. Debe asegurarse que ellos también reciban o puedan conseguir el código fuente. Y debe mostrarles estas condiciones para que conozcan sus derechos.\\

Los desarrolladores que usen la GPL GNU le protegen sus derechos de dos formas: (1) imponen derechos al software, y (2) le ofrecen esta Licencia para que legalmente lo copie, distribuya y/o modifique.\\

Para proteger a desarrolladores y autores, la GPL expone claramente que no existe  garantía  alguna  para  este  software  libre.  Para  beneficio  de  ambos,  usuarios  y autores, la GPL establece que las versiones modificadas deben estar identificadas de forma apropiada como tales, para que cualquier problema no sea atribuido por error
a los autores de versiones anteriores.\\

Algunos dispositivos son diseñados para negar al usuario la instalación o ejecución de versiones modificadas del software que usan internamente, aunque el fabricante sí puede hacerlo. Esto es completamente incompatible con el objetivo de proteger la libertad de los usuarios para modificar el software. Este tipo de abuso sistemático
ocurre con productos de consumo para uso personal, que es precisamente en donde es menos aceptable. Por tanto, hemos diseñado esta versión de la GPL para prohibir estas prácticas en esos productos. Si apareciesen problemas similares en otros ámbitos, estaremos atentos para extender estas prestaciones en las próximas versiones de la GPL, tanto como sea necesario para proteger la libertad de los usuarios.\\

Por último, todo programa está constantemente amenazado por las patentes de software. Los estados no deberán permitir que las patentes restrinjan el desarrollo y  el  uso  de  software  en  ordenadores  de  propósito  general,  pero  en  aquellos  que  lo hagan, esperamos evitar el especial peligro que suponen las patentes, que aplicadas
a un programa libre puedan hacerlo propietario en la práctica. Para prevenir eso, la GPL establece que las patentes no pueden usarse para convertir un programa en no-libre.\\

A continuación siguen los términos precisos y las condiciones para la copia, distribución y modificación.\\

\textbf{Términos y condiciones para la copia, distribución y modificación}

\begin{itemize}
	
	\item En adelante "Esta Licencia" se refiere a la versión 3 de la Licencia Publica General GNU.
	
	\item "Copyright" también significa "leyes similares al copyright" que son aplicables a otro tipo de trabajos, como las mascaras de semiconductores.\\
	
	\item "El Programa" se refiere a cualquier trabajo con copyright al que se haya aplicado esta Licencia. Cada concesionario es asimilable a "usted". "Concesionarios" y	"destinatarios" pueden ser personas físicas u organizaciones.\\
	
	\item "Modificar" un trabajo significa copiar o adaptar en su totalidad o parcialmente	un trabajo de manera que requiera permiso de copyright, de cualquier tipo salvo la copia exacta. El trabajo resultante se denomina "versión modificada" de un trabajo anterior o trabajo "basado en" el trabajo anterior.\\
	
	\item Un "trabajo amparado" puede ser tanto el Programa no modificado como un trabajo basado en el Programa.\\
	
	\item "Difundir" un trabajo significa hacer cualquier cosa con el, sin permiso, que le haga directa o indirectamente responsable de infringir leyes cubiertas por copyright, excepto la ejecución en un ordenador o la modificación de una copia privada. La difusión incluye la copia, distribución (con o sin modificaciones), distribución pública,
	y en algunos países también otras actividades.\\
	
	\item "Transmitir" un trabajo significa cualquier tipo de difusión que permite a la otra parte hacer o recibir copias. La mera interacción con un usuario mediante una red de ordenadores, sin transferir copia alguna, no se considera "transmisión".\\

\end{itemize}

 Una interfaz de usuario interactivo muestra "Avisos Legales Apropiados" siempre	y cuando incluya características visuales apropiadas y destacadas que (1) muestren un aviso de copyright apropiado, y (2) indiquen al usuario que no existe garantía alguna para el trabajo (exceptuando las garantías que se hayan podido establecer), que  los  concesionarios  deben  acompañar  el  trabajo  con  esta  Licencia,  y  cómo  se puede ver una copia de esta Licencia. Si la interfaz muestra una lista de opciones o comandos, tales como menús, un elemento destacado en dicha lista cumple estos criterios.\\

\begin{enumerate}
	
	\item Código Fuente.\\
	
	El "código fuente" de un trabajo es el formato preferido para realizar modificaciones en él. "Código objeto" se refiere a cualquier formato del trabajo que no sea código fuente.\\
	
	Una "Interfaz Estándar" se refiere a una interfaz que sea o bien un estándar oficial definido por una institución de estándares reconocida, o bien, en el caso de interfaces específicas para un determinado lenguaje de programación, uno cuyo uso esté extendido entre los desarrolladores que trabajan con ese lenguaje.\\
	
	Las "Bibliotecas de Sistema" de un trabajo ejecutable incluyen cualquiera, que no sea el trabajo completo, que (a) esté incluida de la misma forma que un componente principal, pero que no forme parte de ese componente principal, y (b) sólo sirva para habilitar la utilización del trabajo a través de ese componente principal, o para implementar una Interfaz Estándar para el cual está disponible una implementación pública en código fuente. Un "Componente Principal", en este contexto, se refiere a un componente principal y esencial (núcleo, sistema de ventanas y similares) del sistema operativo particular (en su caso) sobre el cual funcione el ejecutable, o un compilador utilizado para generar el trabajo, o un intérprete de código objeto utilizado para ejecutarlo.\\
	
	La "Fuente Correspondiente" de un trabajo en código objeto se refiere a todo el código fuente necesario para generar, instalar, y (para un trabajo ejecutable) ejecutar el código objeto y modificar el trabajo, incluyendo guiones que controlen esas actividades. De todas formas, no se incluyen las Bibliotecas de Sistema del trabajo, o herramientas de propósito general o programas gratuitos habitualmente disponibles y usados para realizar estas actividades que no forman parte del trabajo. Por ejemplo, la Fuente Correspondiente incluye los archivos de definición de interfaz asociados con archivos fuente del trabajo, y el código fuente de las bibliotecas compartidas o subprogramas enlazados dinámicamente que el programa requiere por diseño, como la comunicación de datos intrínseca o el control de  flujo entre esos subprogramas y otras partes del trabajo.\\
	
	La Fuente Correspondiente no incluye necesariamente aquello que los usuarios pueden regenerar automáticamente desde otras partes de la Fuente Correspondiente.\\
	
	La Fuente Correspondiente de un trabajo en código fuente es ese mismo trabajo.\\
	
	\item Permisos Básicos.\\
	
	Todos los derechos garantizados por esta Licencia se otorgan como copyright del Programa, y se proporcionan de manera irrevocable siempre y cuando se cumplan las condiciones establecidas. ésta Licencia afirma explícitamente su permiso ilimitado para ejecutar el Programa sin modificaciones. El resultado de la ejecución de un programa amparado está cubierto por esta Licencia sólo si la salida, por su contenido, constituye un trabajo amparado. Esta Licencia reconoce sus derechos por uso razonable u otro equivalente, tal y como determina la ley de copyright.\\
	
	Usted puede realizar, ejecutar y difundir trabajos amparados que no transmite, sin condición alguna mientras no tenga otra licencia más restrictiva. Puede transportar trabajos amparados a terceros con el mero objetivo de que ellos hagan modificaciones exclusivamente para usted, o para que le proporcionen ayuda para ejecutar esos trabajos, siempre que cumpla los términos de esta Licencia transportando todo el material de cuyo copyright no posee el control. Aquellos que realicen o ejecuten los trabajos amparados para usted deben hacerlo exclusivamente en su nombre, bajo su dirección y control, con términos que les prohíban realizar copias de su material con copyright al margen de su relación con usted.\\
	
	El transmisión bajo otras circunstancias se permite únicamente bajo las condiciones establecidas más abajo. No está permitido sublicenciar; la sección 10 lo hace innecesario.\\
	
	\item Protección de Derechos Legales de Usuarios frente a Leyes Anti-Evasión.\\
	
	Ningún trabajo amparado debe considerarse parte de una medida tecnológica efectiva, bajo cualquier ley aplicable que cumpla las obligaciones del artículo 11 del tratado de copyright WIPO adoptado el 20 de diciembre de 1996, o leyes similares que prohíben o restringen la evasión de tales medidas.\\
	
	Cuando transmite un trabajo amparado, usted renuncia a cualquier poder legal para prohibir la evasión de medidas tecnológicas mientras tales evasiones se realicen en ejercicio de derechos amparados por esta Licencia respecto al trabajo amparado, y usted niega cualquier intención de limitar el uso o modificación del trabajo con el objetivo de imponer, al trabajo de los usuarios, sus derechos legales o de terceros para prohibir la evasión de medidas tecnológicas.\\
	
	\item Transmisión de copias literales.\\
	
	Usted puede transmitir copias literales del código fuente del Programa tal y como lo recibe, por cualquier medio, siempre que publique de forma clara y llamativa en cada copia un aviso de copyright apropiado; mantenga intactos todos los avisos que afirmen que esta Licencia y cualquier término no-permisivo añadido y acorde con la sección 7 son aplicables al código; mantenga intactos todos los avisos de ausencia de garantía; y proporcione a todos los destinatarios una copia de esta Licencia junto con el Programa.\\
	
	Usted puede cobrar cualquier importe o no cobrar nada por cada copia que transmite, y puede ofrecer soporte o protección de garantía mediante un pago.\\
	
	\item Transmisión de Versiones Modificadas de Código.\\
	
	Usted puede transmitir un trabajo basado en el Programa, o las modificaciones que lo producen a partir del Programa, como código fuente bajo los términos de la sección 4, siempre que cumpla todas las condiciones siguientes:\\
	
	\begin{enumerate}
		
		\item El trabajo debe incluir avisos destacados indicando que usted lo ha modificado, dando una fecha pertinente.\\
		
		\item El trabajo debe incluir avisos destacados indicando que está realizado bajo esta Licencia y cualquier otra condición añadida bajo la sección 7.\\
		
		\item Usted debe aplicar la licencia al trabajo completo, como un todo, a cualquier persona que tenga posesión de una copia. Esta Licencia se aplicará por tanto, al igual que cualquier otra condición adicional de la sección 7, al conjunto completo del trabajo y todas y cada una de sus partes, independientemente de como sean agrupadas o empaquetadas. Esta Licencia no permite ser aplicada al trabajo de ninguna otra forma, pero no se anula dicho permiso si usted lo ha recibido por separado.\\
		
		\item Si el trabajo tiene interfaces de usuario interactivas, cada uno debe mostrar Avisos Legales Apropiados; de todas formas, si el Programa tiene interfaces interactivas que no muestran Avisos Legales Apropiados, su trabajo no tiene porqué modificarlos para que lo hagan.\\
		
	\end{enumerate}
	
	Un conjunto o recopilación formado por un trabajo amparado y otros trabajos distintos e independientes, que por su naturaleza no sean extensiones del trabajo amparado, y que no se combinen con él de alguna forma para dar lugar a un programa mayor, ubicados en un medio de distribución o almacenamiento, es llamado un "paquete" si la recopilación y su copyright al completo no son usados para limitar los derechos legales de los usuarios de la recopilación, más allá de lo que el trabajo individual permita. La inclusión de un trabajo amparado en un paquete no hace aplicable esta Licencia al resto de elementos del paquete.\\
	
	\item Transmisión de código No-fuente.\\
	
	Usted puede transmitir el código objeto de un trabajo amparado bajo los términos de las secciones 4 y 5 siempre que también transmite las Fuentes Correspondientes en lenguaje máquina bajo los términos de esta Licencia de alguna de las siguientes maneras:\\
	
	\begin{enumerate}
		
		\item Transmitir el código objeto en, o incorporado en, un producto físico (incluyendo medios de distribución físicos), acompañado por las Fuentes Correspondientes en un medio físico duradero habitual para el intercambio de software.\\
		
		\item Transmitir el código objeto en, o incorporado en, un producto físico (incluyendo medios de distribución físicos), acompañado de una oferta por escrito, válida al menos durante tres años y válida durante el tiempo que usted ofrezca recambios o soporte para ese modelo de producto, que ofrezca al poseedor del código objeto (1) una copia de las Fuentes Correspondientes a todo el software del producto que esté cubierto por esta Licencia, en un medio físico duradero habitual para el intercambio de software, a un precio no mayor que su coste razonable por transmitir físicamente las fuentes, o (2) acceder a copiar las fuentes correspondientes desde un servidor de red sin coste alguno.\\
		
		\item Transmitir copias individuales del código objeto junto a una copia de una oferta por escrito para proporcionar las Fuentes Correspondientes. Esta alternativa sólo está permitida ocasional y no comercialmente, y solamente si usted recibió el código objeto junto a una oferta parecida, de acuerdo con la subsección 6b.\\
		
		\item Transmitir el código objeto ofreciendo acceso desde un lugar determinado (gratuitamente o mediante pago), y ofrecer acceso equivalente a las Fuentes Correspondientes de la misma forma y el mismo lugar sin cargo añadido. No es necesario requerir a los destinatarios copiar las Fuentes Correspondientes junto al código objeto. Si el lugar para copiar el código objeto es un servidor de red, las Fuentes Correspondientes pueden estar en un servidor diferente (gestionado por usted o tercero) que ofrezca facilidades de copia equivalentes, siempre que mantenga instrucciones claras junto al código objeto sobre dónde encontrar las Fuentes Correspondientes. Independientemente de qué servidores alberguen las Fuentes Correspondientes, usted sigue obligado a asegurar que estarán disponibles durante el tiempo necesario para cumplir estos requisitos.\\
		
		\item Transmitir el código mediante transferencias entre usuarios, siempre que informe a otros usuarios dónde se ofrecen el código objeto y las Fuentes Correspondientes de forma pública sin cargo alguno bajo tal y como establecer la subsección 6d. Una parte separable del código objeto, cuyo código fuente esté excluido de las Fuentes Correspondientes como Biblioteca de Sistema, no necesita ser incluida en el transmisión del código objeto del trabajo.\\
		
	\end{enumerate}
	
	Un "Producto de Usuario" es tanto (1) un "producto de consumo", que se refiere a cualquier propiedad personal tangible habitualmente utilizada para fines personales, familiares o domésticos, o (2) cualquier cosa diseñada o vendida para ser incorporada como extensión. Para determinar si un producto es un producto de consumo, los casos dudosos se resolverán favoreciendo el amparo. Para un producto en particular recibido por un usuario particular, "de uso habitual" se refiere al uso típico o corriente de ese tipo de producto, independientemente de la situación del usuario particular o de la forma en que el usuario particular utilice, o pretenda o se espere que pretenda utilizar, el producto. Un producto es un producto de consumo independientemente de si el producto tiene usos sustancialmente comerciales, industriales o no de consumo, a menos que tales usos representen el único tipo significativo de uso del producto.\\
	
	La "Información de Instalación" para un Producto de Usuario se refiere a cualquier método, procedimiento, clave de autorización, u otro tipo de información necesaria para instalar y ejecutar una versión modificada de un trabajo amparado en ese Producto de Usuario a partir de una versión modificada de las Fuentes Correspondientes. La información debe ser suficiente para asegurar el funcionamiento continuo del código objeto modificado sin ningún tipo de condicionamiento o intromisión por el simple hecho de haber sido modificado.\\
	
	Si usted transmite bajo las premisas de esta sección el código objeto de un trabajo en, o con, o específicamente para ser usado en un Producto de Usuario, y el transmisión forma parte de una transacción donde los derechos de posesión y uso del Producto de Usuario se transfieren al destinatario a perpetuidad o durante un plazo fijo de tiempo (independientemente de las características de la transacción), las Fuentes Correspondientes transmitidas bajo estos supuestos debe ser acompañada de la Información de Instalación. Estos requerimientos no se aplican si ni usted ni terceros tienen posibilidad de instalar código objeto modificado en el Producto de Usuario (por ejemplo, el trabajo ha sido instalado en memoria de sólo lectura, ROM):\\
	
	El requerimiento de proporcionar Información de Instalación no incluye el requerimiento de continuar proporcionando servicio de soporte, garantía, o actualizaciones para un trabajo que haya sido modificado o instalado por el destinatario, o para el Producto de Usuario en el que se haya modificado o instalado.\\
	
	El acceso a la red puede ser denegado cuando la propia modificación afecte material y adversamente a la operación de la red o viole las reglas y protocolos de comunicación en la red. Las Fuentes Correspondientes transmitidas, y las Instrucciones de Instalación proporcionadas de acuerdo con esta sección, deben figurar en un formato documentado públicamente (y con una implementación disponible para el público en código fuente), y no debe necesitar claves de acceso especiales para la descompresión, lectura o copia.\\
	
	\item Condiciones adicionales.\\
	
	"Permisos Adicionales" son condicionantes que amplían los términos de esta Licencia permitiendo excepciones a una o más de sus condiciones. Los Permisos Adicionales que son aplicables al Programa completo deberán ser tratados como si estuviesen incluidos en esta Licencia, hasta los límites de validez impuestos por las leyes aplicables. Si los permisos adicionales se aplicasen sólo a una parte del Programa, esa parte podría ser usada de forma independiente bajo esos permisos, pero el Programa completo seguiría estando afectado por esta Licencia con independencia de los permisos adicionales.\\
	
	Cuando transmite una copia de un trabajo amparado, puede opcionalmente eliminar cualquier permiso adicional de esa copia, o de alguna parte del mismo. (Los permisos adicionales pueden haber establecido que sea requerida su eliminación en ciertos supuestos si usted modifica el trabajo.) Usted puede establecer permisos adicionales en material añadido por usted a un trabajo amparado, sobre el cual tiene o puede aportar sus permisos de copyright correspondientes. Con independencia de cualquier otra estipulación en esta Licencia, usted puede, para el material que añada a un trabajo amparado, (si está autorizado por los poseedores de copyright de ese material) añadir condiciones a esta Licencia con los siguientes términos:\\
	
	\begin{enumerate}
		
		\item Ausencia de garantía o limitación de responsabilidad diferente a los términos de las secciones 15 y 16 de esta Licencia; u\\
		
		\item Obligación de mantener determinados avisos legales razonables o atribuciones de autoría en el material o en los Avisos Legales Correspondientes mostrados por los trabajos que lo contengan; o\\
		
		\item Prohibir la tergiversación del origen del material, o solicitar que las diferencias respecto a la versión original sean advertidas de forma apropiada en las versiones modificadas del material; o\\
		
		\item Limitar la utilización de los nombres de los autores o licenciatarios del material con fines divulgativos ; o\\
		
		\item Negarse a ofrecer derechos afectados por leyes de registro para el uso de marcas empresariales, registradas o de servicio; o\\
		
		\item Requerir indemnización a los autores y poseedores de licencia de ese material, por parte de cualquier persona que transmite el material (o versiones modificadas del mismo) estableciendo obligaciones contractuales de responsabilidad sobre el destinatario, para cualquier responsabilidad que estas obligaciones contractuales impongan directamente sobre los autores y poseedores de licencia.\\
		
		\end {enumerate}
		
		Cualquiera otras condiciones adicionales no-permisivas son consideradas "restricciones posteriores" en el contexto de la sección 10. Si el Programa, tal cual lo recibió, o cualquier parte del mismo, contienen un aviso indicando que está amparado por esta Licencia junto a una cláusula de restricción posterior, usted puede suprimir esa cláusula. Si un documento de licencia contiene una restricción posterior pero permite modificar la licencia o el transmisión bajo esta Licencia, usted puede añadirla al material de un trabajo amparado por los términos de ese documento de licencia, siempre que la restricción posterior no se mantenga tras la modificación de la licencia o el transmisión.\\
		
		Si añade condiciones acordes a esta sección para un trabajo amparado, usted debe ubicar, en los archivos fuente involucrados, una declaración de los términos adicionales aplicables a esos archivos, o un aviso indicando dónde localizar los términos aplicables.\\
		
		Condiciones adicionales, permisivas o no, deben aparecer por escrito como licencias separadas, o figurar como excepciones; de todas formas los requisitos anteriores siempre son aplicables.\\
		
		\item Cancelación.\\
		
		Usted no puede transmitir o modificar un trabajo amparado salvo como expresamente se ha previsto en esta Licencia. Cualquier intento diferente de transmisión o modificación será considerado nulo, y automáticamente cancelará sus derechos respecto a esta Licencia (incluyendo cualquier patente conseguida según el párrafo tercero de la sección 11).\\
		
		Sin embargo, si deja de violar esta Licencia, entonces su licencia desde el poseedor del copyright correspondiente será restituida (a) provisionalmente, a menos que y hasta que el poseedor del copyright de por terminada explícita y permanentemente su licencia, y (b) permanentemente, si el poseedor del copyright no le ha notificado por algún cauce de la violación en los 60 días posteriores al cese. Además, su licencia desde el poseedor del copyright correspondiente será restituida permanentemente si el poseedor del copyright le notifica de la violación por algún cauce, es la primera vez que recibe la notificación de violación de esta Licencia (para cualquier trabajo) de ese poseedor de copyright, y usted subsana la violación antes de 30 días desde la recepción del aviso.\\
		
		La cancelación de sus derechos según esta sección no da por canceladas las licencias de terceros que hayan recibido copias o derechos a través de usted con esta Licencia. Si sus derechos han finalizado y no han sido restituidos de forma permanente, usted no está capacitado para recibir nuevas licencias para el mismo material según la sección 10.\\
		
		\item Aceptación no necesaria por tenencia de copias.\\
		
		No es necesario aceptar esta Licencia para recibir o ejecutar una copia del Programa. La distribución de un trabajo amparado surgida simplemente como consecuencia de usar transmisión entre usuarios para obtener una copia tampoco requiere aceptación. De todas formas, no puede usar otra salvo esta Licencia para transmitir o modificar cualquier trabajo amparado.\\
		
		\item Transmisión automática de licencia para destinatarios Cada vez que transmite un trabajo amparado, el destinatario recibirá automáticamente una licencia desde los poseedores originales, para ejecutar, modificar y transmitir ese trabajo, objeto de esta Licencia. Usted no es responsable de asegurar el cumplimiento por terceros de esta Licencia.\\
		
		Una "transacción de entidad" es una transacción que transfiere el control de una organización, o todos los bienes sustanciales de una, o subdivide una organización, o fusiona organizaciones. Si el transmisión de un trabajo amparado surge de una transacción de entidad, cada parte involucrada en esa transacción que reciba una copia del trabajo, también recibe cualquier licencia existente del trabajo cuya parte interesada tuviese o pudiese ofrecer según el párrafo anterior, además del derecho a tomar posesión de las Fuentes Correspondientes del trabajo a través de la parte interesada, si está en poder de dicha parte o se puede conseguir con un esfuerzo razonable.\\
		
		Usted no puede imponer restricciones posteriores en el ejercicio de los derechos otorgados o concedidos por esta Licencia. Por ejemplo, usted no puede imponer a la licencia pagos, derechos o otros cargos por el ejercicio de los derechos otorgados según esta Licencia, y usted no puede iniciar litigios (incluyendo demandas o contra demandas en pleitos) alegando que se infringen patentes por cambiar, usar, vender, ofrecer en alquiler o importar el Programa, o cualquier parte del mismo.\\
		
		\item Patentes.\\
		
		Un "colaborador" es un poseedor de copyright que autoriza el uso bajo esta Licencia del Programa o un trabajo en el que se base el Programa. El trabajo con esta licencia se denomina "versión en colaboración" con el colaborador. Todas las reivindicaciones de patentes en posesión o controladas por el colaborador se denominan "demandas de patente original", ya sean existentes o adquiridas, que hayan sido infringidas de alguna forma permitida por esta Licencia, al hacer, usar o vender la versión en colaboración, pero sin incluir demandas que sólo sean infracciones como consecuencia de modificaciones posteriores de la versión en colaboración. Para aclarar esta definición, "control" incluye el derecho de conceder sublicencias de patente según los requerimientos de esta Licencia.\\
		
		Cada colaborador le concede a usted una licencia de la patente no-exclusiva, global y libre de derechos bajo las reivindicaciones de patente de origen del colaborador, para el uso, modificación, venta, ofertas de venta, importación y otras formas de ejecución, modificación y redistribución del contenido de la versión en colaboración.\\
		
		En los siguientes tres párrafos, una "licencia de patente" se refiere a cualquier expresión de acuerdo o compromiso, independientemente de la denominación, que no imponga una patente (como puede ser el permiso expreso para ejecutar una patente o acuerdos para no imponer demandas por infracción de patente). "Conceder" estas licencias de patente a un tercero significa llegar a tal tipo de acuerdo o compromiso que no imponga una patente al tercero.\\
		
		Si usted transmite un trabajo amparado, conociendo que está afectado por licencia de patente, y no están disponibles de forma pública para su copia las Fuentes Correspondientes, sin cargo alguno y bajo los términos de esta Licencia, ya sea a través de un servidor de red público o mediante cualquier otro medio, entonces usted debe o (1) permitir que sean públicas las Fuentes Correspondientes, o (2) tratar de eliminar los beneficios de la licencia de patente para este trabajo en particular, o (3) tratar de extender, de forma adecuada a los requisitos de esta Licencia, la licencia de patente a terceros. "Conocer que está afectado" significa que usted tiene conocimiento actual de que, para la licencia de patente, el transmisión del trabajo amparado en un determinado país, o el uso del trabajo amparado por sus destinatarios en un determinado país, infringiría una o más patentes existentes en ese país que usted considera aplicables por algún motivo. Si para conseguir una transacción o acuerdo(o en un proceso relacionado con ellos), usted transmite o distribuye con fines de transmisión , un trabajo amparado, concediendo una licencia de patente para algún tercero que reciba el trabajo amparado, y autorizándole a usar, distribuir, modificar o transmitir una copia específica del trabajo amparado, entonces la licencia de patente que usted otorgue se extiende automáticamente a todos los receptores del trabajo amparado y cualquier trabajo basado en el mismo.\\
		
		Una licencia de patente es "discriminatoria" si no incluye dentro de su ámbito de cobertura, prohíbe el ejercicio, o está condicionada a no ejercitar uno o más de los derechos que están específicamente otorgados por esta Licencia. Usted no debe transmitir un trabajo amparado si está implicado en un acuerdo con terceros que estén relacionados con el negocio de la distribución de software, en el que usted haga pagos relacionados con su actividad del transmisión del trabajo, y donde se otorgue, a cualquier receptor del trabajo amparado, una licencia de patente discriminatoria (a) en relación con las copias de trabajo amparado transmitido por usted (o copias hechas desde éstas), o (b) directa o indirectamente relacionadas con productos específicos o paquetes que contengan el trabajo amparado, a menos que usted forme parte del acuerdo, o esa licencia de patente fuese otorgada antes del 28 de marzo de 2007.\\
		
		Ninguna disposición de esta Licencia debe ser considerada como excluyente o limitante de la aplicación de cualquier otra licencia o defensas legales contra la violación de las leyes de propiedad intelectual a que pudiera tener derecho bajo la ley de propiedad intelectual vigente.\\
		
		\item No condicionamiento de la libertad de terceros.\\
		
		Si a usted le son aplicables condiciones que contradicen las condiciones de esta Licencia (ya sea por orden judicial, acuerdo u otros), no queda eximido de cumplir las condiciones de esta Licencia. Si usted no puede transmitir un trabajo amparado cumpliendo simultáneamente sus obligaciones con esta Licencia y con cualquier otra pertinente, entonces no podrá transmitirlo de ninguna forma. Por ejemplo, si usted se compromete con términos que le obligan a obtener derechos por el transmisión a terceros, la única forma de satisfacer ambos condicionantes y esta Licencia es abstenerse completamente de transmitir el Programa.\\
		
		\item Uso conjunto con la Licencia Pública General Affero GNU.\\
		
		Con independencia de cualquier disposición en esta Licencia, usted tiene permiso para enlazar o combinar cualquier trabajo amparado con otro trabajo amparado por la versión 3 de la Licencia Pública General Affero GNU, para formar un solo trabajo combinado, y transmitir el trabajo resultante. Los términos de esta Licencia seguirán siendo aplicables a la parte formada por el trabajo amparado, pero los condicionantes especiales de la Licencia Pública General Affero GNU, en su sección 13, relativos a la interacción mediante redes, serán aplicables a la combinación de ambas partes.\\
		
		\item Versiones Revisadas de esta Licencia.\\
		
		La Fundación para el Software Libre puede publicar revisiones y/o nuevas versiones de la Licencia Pública General GNU de vez en cuando. Esas versiones serán similares en espíritu a la versión actual, pero podrán diferir en algunos detalles para afrontar nuevos problemas o situaciones.\\
		
		A cada versión se le da un número distintivo. Si el Programa especifica que le es aplicable cierto número de versión de la Licencia Pública General o "cualquier versión posterior", usted tiene la posibilidad de adoptar los términos y condiciones de la versión indicada o de cualquier otra versión posterior publicada por la Fundación para el Software Libre. Si el Programa no especifica un número de versión de la Licencia Pública General, usted puede elegir cualquier versión que haya sido publicada por la Fundación para el Software Libre.\\
		
		Si el Programa especifica que un apoderado puede decidir qué versiones de la Licencia Pública General pueden aplicarse en el futuro, la declaración pública de aceptación que el apoderado haga de una versión le autoriza a usted con carácter permanente a elegir esa versión para el Programa.\\
		
		Versiones de licencia posteriores pueden otorgarle permisos adicionales o diferentes.\\
		
		De todas formas, no pueden imponerse obligaciones adicionales a cualquier autor o poseedor de copyright como consecuencia de su elección de adoptar una versión posterior.\\
		
		\item Ausencia de Garantía.\\
		
		EL PROGRAMA NO TIENE GARANTÍA ALGUNA, HASTA LOS LÍMITES PERMITIDOS POR LAS LEYES APLICABLES SALVO CUANDO SE ESTABLEZCA LO CONTRARIO POR ESCRITO EL POSEEDOR DEL COPYRIGHT Y/O TERCEROS PROPORCIONAN EL PROGRAMA "TAL CUAL" SIN GARANTÍA DE NINGÚN TIPO, YA SEA EXPLÍCITA O IMPLÍCITA, INCLUYENDO, PERO SIN LIMITARSE A, LAS GARANTÍAS IMPLÍCITAS MERCANTILES Y DE APTITUD PARA UN PROPÓSITO DETERMINADO. USTED ASUME CUALQUIER RIESGO RELATIVO A LA CALIDAD Y RENDIMIENTO DEL PROGRAMA. SI EL PROGRAMA FUESE DEFECTUOSO, USTED ASUME EL COSTE DE CUALQUIER COSTE DE SERVICIO, REPARACIÓN O CORRECCIÓN.\\
		
		\item Limitación de Responsabilidad\\
		
		EN NINGÚN CASO, SALVO REQUERIMIENTO POR LEYES APLICABLES O ACUERDO POR ESCRITO, PODRÁ UN POSEEDOR DE COPYRIGHT, O UN TERCERO QUE MODIFIQUE O TRANSMITE EL PROGRAMA SEGÚN LO INDICADO ANTES, HACERLE A USTED RESPONSABLE DE DAÑO ALGUNO, INCLUYENDO CUALQUIER DAÑO GENERAL, ESPECIAL, OCASIONAL O DERIVADO QUE SURJA DEL USO O LA INCAPACIDAD DE USO DEL PROGRAMA (INCLUYENDO PERO SIN LIMITARSE A LA PÉRDIDA DE DATOS O LA PRESENTACIÓN NO PRECISA DE DATOS O PÉRDIDAS SUFRIDAS POR USTED O TERCEROS O FALLO DEL PROGRAMA AL INTERACTUAR CON OTROS PROGRAMAS), INCLUSO SI EL POSEEDOR O UN TERCERO HA SIDO ADVERTIDO DE LA POSIBILIDAD DE TALES DAÑOS.\\
		
		\item Interpretación de la sección "Ausencia de Garantía".\\
		
		Si la ausencia de garantía y la limitación de responsabilidad descrita anteriormente no tuviesen efecto legal a nivel local en todos sus términos, los juzgados aplicarán las leyes locales que más se aproximen a la exención de responsabilidad civil en lo relativo al Programa, a menos que la copia del Programa esté acompañada mediante pago de una garantía o compromiso de responsabilidad.\\
		
	\end{enumerate}
	

%GNU \input{./anexos/fdl-1.3}

\end{document}
