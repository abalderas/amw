% ------------------------------------------------------------------------------
% Este fichero es parte de la plantilla LaTeX para la realización de Proyectos
% Final de Grado, protegido bajo los términos de la licencia GFDL.
% Para más información, la licencia completa viene incluida en el
% fichero fdl-1.3.tex

% Copyright (C) 2012 SPI-FM. Universidad de Cádiz
% ------------------------------------------------------------------------------

\thispagestyle{empty}

\noindent \textbf{\begin{Large}Resumen\end{Large}} 
\newline
\newline
\noindent Los wikis son un sistema muy popular como ayuda a la docencia. Cuando el número de estudiantes y la cantidad de información almacenada en un wiki aumentan, evaluar el trabajo de cada estudiante resulta difícil. Los wikis mantienen un registro con las diferencias entre las revisiones consecutivas de los artículos, que pueden ser usadas para la evaluación del aprendizaje. Esta información puede evaluarse a lo largo de la vida del wiki para obtener datos sobre la actividad de los estudiantes.
\newline

AssessMediaWiki es una aplicación web de código abierto que, al conectarse a una instalación MediaWiki, proporciona procedimientos de autoevaluación, hetero evaluación y evaluación entre iguales, a la vez que mantiene información sobre esas evaluaciones. Los supervisores pueden obtener informes que ayudan en la evaluación de los estudiantes.
\newline

Aunque hay un gran número de extensiones para el sistema MediaWiki, no hemos encontrado ninguna que permitiera evaluar contribuciones individuales a un wiki. La mayoría de las aproximaciones solo ofrecen formas de evaluar una versión en particular de un artículo (normalmente la más reciente), siendo ineficaces en este caso. Por ello, para evaluar la calidad de las contribuciones creamos AssessMediaWiki.
\newline

AssessMediaWiki implementa como base dos roles de usuario distintos: supervisores y estudiantes. Los estudiantes pueden elegir entre distintas opciones: evaluar una revisión, comprobar sus propias aportaciones evaluadas y verificar las evaluaciones ya enviadas. Por otro lado, los supervisores tienen un mayor número de opciones, como modificar los parámetros de los programas o vigilar las evaluaciones que los alumnos vayan haciendo.
\newline

\noindent {\bf Palabras clave:} AssessMediaWiki, MediaWiki, Wiki, software libre, evaluación
