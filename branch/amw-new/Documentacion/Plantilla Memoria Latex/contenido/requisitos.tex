% ------------------------------------------------------------------------------
% Este fichero es parte de la plantilla LaTeX para la realización de Proyectos
% Final de Grado, protegido bajo los términos de la licencia GFDL.
% Para más información, la licencia completa viene incluida en el
% fichero fdl-1.3.tex

% Copyright (C) 2012 SPI-FM. Universidad de Cádiz
% ------------------------------------------------------------------------------


\section{Situación actual} 
En la versión actual (la que está funcionando en estos momentos) de AssessMediaWiki nos encontramos con un algoritmo de selección que escoge una edición de forma aleatoria entre las $n$ ediciones mas significativas (las de mayor tamaño), de forma que en el peor de los casos nos encontramos con la situación de que las $n-1$ ediciones mas significativas nunca serán seleccionadas.
\newline
Esta versión de AssessMediaWiki también carece de un procesado de ediciones para ciertos wiki-comportamientos (ráfagas, correcciones ortográficas, etc) y los roles de los usuarios se limitan a los de alumno y supervisor.\\
Estos situaciones han sido complementadas con esta nueva versión, añadiendo un algoritmo distinto de selección de ediciones, un sistema de administración de roles y el procesado de ráfagas

\subsection{Entorno Tecnológico}
Para este proyecto sera necesario contar con un MediaWiki, normalmente alojado en un servidor, en el cual seria recomendable alojar el mismo software del proyecto.

\subsection{Fortalezas y Debilidades}
Fortalezas: Herramienta muy útil para añadir nuevas posibilidades a los métodos docentes.//

Debilidades: Es necesario un mínimo de conocimientos para su instalación, al igual que para la instalación del MediaWiki.

\section{Objetivos del Sistema}
El objetivo del sistema es añadir nuevas funciones al método docente y darle a los alumnos mas responsabilidades, así como analizar su participación y resultados.

\section{Catálogo de Requisitos}

\subsection{Requisitos funcionales}
Listado de las nuevas funcionalidades que ofrece el sistema:

\begin{itemize}
	\item Creación y edición de nuevos roles.
	\item Creación y edición de ejercicios de evaluación.
	\item Procesado de ráfagas.
	\item Preasignaciones de ediciones.
\end{itemize}

\subsection{Requisitos no funcionales}
Descripción de otros requisitos (relacionados con la calidad del software) que el sistema deberá satisfacer: portabilidad, seguridad, estándares de obligado cumplimiento, accesibilidad, usabilidad, etc:
\begin{itemize}
	\item MediaWiki.
	\item Servidor para hospedar tanto el MediaWiki como AssessMediaWiki.
	\item Conexión a internet.
\end{itemize}

\subsection{Requisitos de información}
En esta sección se describen los requisitos de gestión de información (datos) que el sistema debe gestionar:
\begin{itemize}
	\item IDs de los usuarios del MediaWiki.
	\item IDs de las ediciones creadas por los usuarios.
	\item Notas y comentarios asignados a ediciones evaluadas.
	\item Parámetros de configuración
	\item Roles creados por el administrador.
	\item Ejercicios de evaluación y sus configuraciones.
\end{itemize}

\section{Alternativas de Solución}
Se presento la posibilidad de usar \href{http://evalcomix.uca.es/}{evalcomix} \cite{Evalcomix} como herramienta de evaluación alternativa.

\section{Solución Propuesta}
Se consideró mejor opción implementar un método propio que fuese escalable, por lo que se descarto usar \href{http://evalcomix.uca.es/}{evalcomix} \cite{Evalcomix} en el desarrollo del sistema.
\newline

Se ha implementado un sistema de evaluación en el que el docente puede configurar cuantas ediciones se evaluaran de cada estudiante y cuantas evaluaciones recibirá cada edición.