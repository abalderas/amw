% ------------------------------------------------------------------------------
% Este fichero es parte de la plantilla LaTeX para la realización de Proyectos
% Final de Grado, protegido bajo los términos de la licencia GFDL.
% Para más información, la licencia completa viene incluida en el
% fichero fdl-1.3.tex

% Copyright (C) 2012 SPI-FM. Universidad de Cádiz
% ------------------------------------------------------------------------------

\section{Motivación}
Tras trabajar durante un breve periodo de tiempo en AssessMediaWiki gracias a la beca Icaro, se planteo la posibilidad de seguir trabajando sobre el mismo, pero esta vez para usarlo como proyecto de fin de carrera.
\newline

Tras esa experiencia, viendo todo lo que había prendido de ella (manejar nuevos lenguajes y herramientas de desarrollo, distintas formas de enfocar los problemas, etc), mi interés por aprender y explorar diversos métodos didácticos y el interés y respeto que tengo hacia el software libre y su comunidad, decidí continuar desarrollando AssessMediaWiki y aprovechar para usarlo como proyecto de fin de carrera, con la posterior motivación extra de intentar hacer publicaciones con el.\\

A medida que desarrollaba el proyecto Manuel Palomo me paso varios artículos de investigación que encontré muy interesantes, tanto por las metodologías aplicadas como por la integración con AssessMediaWiki como es poder medir la participación de los usuarios en las wikis a través de sus sesiones \cite{Stuart} .\\

También me paso otro articulo sobre como afecta visualmente a los usuarios la interfaz y como desarrollar herramientas para mejorar este aspecto en las wikis \cite{Mohamad} lo cual me resulto muy interesante ya que el aspecto visual es uno de los que considero mis puntos débiles.\\

A todo eso también debemos añadirle el privilegio de poder trabajar sobre una de las 10 wikis universitarias mas usadas de España \cite{Ortega} .\\

\section{Alcance} 

Este proyecto está orientado a añadir nuevas herramientas y funciones a AssessMediaWiki, dando un abanico mas amplio de posibilidades a los docentes a la hora de interactuar con los alumnos.

\section{Glosario de Términos} 

\begin{itemize}
	\item AMW - AssessMediaWiki
	\item Edición - Aportación realizada por un alumno al MediaWiki
	\item Metaevaluación - evaluación realizada sobre una evaluación existente, para poder ver así la corrección de la evaluación existente
\end{itemize}

\section{Organización del documento}

En el documento se presenta primero la planificación del desarrollo del sistema, así como la metodología de trabajo a seguir.
\newline

En el apartado de desarrollo se muestran un análisis de requisitos y de el sistema actual. También se mesta el proceso para la implementación de las nuevas funciones, así como las modificaciones en las bases de datos.
\newline

En la parte final del documento encontramos el epilogo con el manual de instalación, el manual de uso y la bibliografía utilizada en este documento.



