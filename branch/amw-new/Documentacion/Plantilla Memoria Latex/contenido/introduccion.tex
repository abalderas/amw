% ------------------------------------------------------------------------------
% Este fichero es parte de la plantilla LaTeX para la realización de Proyectos
% Final de Grado, protegido bajo los términos de la licencia GFDL.
% Para más información, la licencia completa viene incluida en el
% fichero fdl-1.3.tex

% Copyright (C) 2012 SPI-FM. Universidad de Cádiz
% ------------------------------------------------------------------------------

\section{Motivación}
Tras trabajar durante un breve periodo de tiempo en AssessMediaWiki gracias a la beca Icaro, se planteo la posibilidad de seguir trabajando sobre el mismo, pero esta vez para usarlo como proyecto de fin de carrera.
\newline

Tras esa experiencia, viendo todo lo que había prendido de ella (manejar nuevos lenguajes y herramientas de desarrollo, distintas formas de enfocar los problemas, etc), mi interés por aprender y explorar diversos métodos didácticos y el interés y respeto que tengo hacia el software libre y su comunidad, decidí continuar desarrollando AssessMediaWiki y aprovechar para usarlo como proyecto de fin de carrera, con la posterior motivación extra de intentar hacer publicaciones con el.

\section{Alcance} 

Este proyecto esta orientado a actualizar los métodos docentes, añadiendo nuevas herramientas y formas de interactuar con los alumnos, empezando por una forma de trabajo en grupo a la que ya puedan estar acostumbrados y permitiendoles tener la responsabilidad de evaluarse entre ellos (no todo el trabajo, solo las partes mas significativas y teniendo el docente la voz final) y a su vez poder ver el progreso individual en las tareas asignadas con la opción de poder obtener estadísticas gracias a programas como StatMediaWiki o CleverFigures, que complementan la experiencia docente 2.0

\section{Glosario de Términos} 

\begin{itemize}
	\item AMW - AssessMediaWiki
	\item Edición - Aportacion realizada por un alumno al MediaWiki
	\item Metaevaluación - evaluación realizada sobre una evaluación existente, para poder ver así la corrección de la evaluación existente
\end{itemize}

\section{Organización del documento}

-Introducción: Resumen de la descripción del proyecto, objetivos y estructura del documento presente.

to do



