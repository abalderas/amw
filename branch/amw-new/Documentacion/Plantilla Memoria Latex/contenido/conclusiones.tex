% ------------------------------------------------------------------------------
% Este fichero es parte de la plantilla LaTeX para la realización de Proyectos
% Final de Grado, protegido bajo los términos de la licencia GFDL.
% Para más información, la licencia completa viene incluida en el
% fichero fdl-1.3.tex

% Copyright (C) 2012 SPI-FM. Universidad de Cádiz
% ------------------------------------------------------------------------------

\section{Objetivos alcanzados}
Este apartado debe resumir los objetivos generales y específicos alcanzados, relacionándolos con todo lo descrito en el capítulo de introducción.\\

Se ha podido ampliar las funcionalidades del sistema AssessMediaWiki, añadiendo mas herramientas para los docentes y mas funciones para los alumnos, con la posibilidad de darles responsabilidades personalizadas y haciendo que la experiencia docente sea mas interactiva entre ellos y con el profesor.

\section{Lecciones aprendidas}
A continuación, se detallan las buenas prácticas adquiridas, tanto tecnológicas como procedimentales, así como cualquier otro aspecto de interés.\\

Resumir cuantitativamente el tiempo y esfuerzo dedicados al proyecto a lo largo de su desarrollo que escribir un sencillo 'he trabajado mucho en este proyecto'.
Tras unos nueve meses (a parte de experiencias previas) dedicándole a este proyecto una media de dos horas diarias (o al menos intentándolo) me he dado cuenta de que realmente lanzar una actualización o una nueva versión de algo lleva mucho tiempo y esfuerzo.\\

También he aprendido mucho sobre los framework, buscar recursos por internet y la ayuda de la comunidad en webs como StackOverflow

\section{Trabajo futuro}
En esta sección, se presentan las diversas áreas u oportunidades de mejora detectadas durante el desarrollo del proyecto y que podrán ser abarcadas en futuras versiones del software.\\

Como trabajo futuro quedan pendiente:
\begin{itemize}
	\item Detección de mas wiki-comportamientos, como las ediciones de correcciones ortográficas y desplazamientos de texto.
	\item Realizar preasignaciones se realicen de forma automática (intentando mantener la independencia con el sistema operativo).
	\item Poder trabajar con AssessMediaWiki vía API.
	\item Integración como una extensión de MediaWiki.
	\item Botón de conflicto de interés / rechazar evaluación.
	\item Estudio de la aplicación de técnicas de Learning Analytics a los resultados cualitativos obtenidos. \cite{Conde} 
\end{itemize}