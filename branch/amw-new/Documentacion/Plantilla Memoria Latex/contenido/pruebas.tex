% ------------------------------------------------------------------------------
% Este fichero es parte de la plantilla LaTeX para la realización de Proyectos
% Final de Grado, protegido bajo los términos de la licencia GFDL.
% Para más información, la licencia completa viene incluida en el
% fichero fdl-1.3.tex

% Copyright (C) 2012 SPI-FM. Universidad de Cádiz
% ------------------------------------------------------------------------------

\section{Estrategia}
Para realizar las pruebas de sistema se ha creado un MediaWiki local con varios usuarios y ediciones.

\section{Entorno de Pruebas}
Para realizar pruebas en el sistema es necesario al menos un MediaWiki con usuarios y ediciones y que se haya creado la propia base de datos del sistema.

\section{Roles}
Describir en esa sección cuáles serán los perfiles y participantes necesarios para la ejecución de cada uno de los niveles de prueba.\\

Los roles de pruebas básicos son el de Estudiante y Profesor, pero dado que se pueden crear roles personalizables en el sistema es imposible hacer un listado total de los roles.

\section{Niveles de Pruebas}
En este sección se documentan los diferentes tipos de pruebas que se han llevado a cabo, ya sean manuales o automatizadas mediante algún software específico de pruebas.\\

Todas las pruebas de inserción de datos han sido realizadas manualmente, así como la corrección de ediciones y la creación y edición tanto de roles como de ejercicios de evaluación

\subsection{Pruebas de Sistema}
Se creo un apartado: "sala de depurado (debug room)" para realizar pruebas preliminares introduciendo datos manualmente en el código en vez de hacerlas con la interfaz, lo cual agilizo mucho el proceso de realización de pruebas, permitiendo hacer varias pruebas de forma simultanea en el sistema.

\subsection{Pruebas de Aceptación}
El objetivo de estas pruebas es demostrar que el producto está listo para el paso a producción. Suelen ser las mismas pruebas que se realizaron anteriormente pero en el entorno de producción. En estas pruebas, es importante la participación del cliente final.\\

Al realizar las pruebas con la interfaz se vieron los mismos errores que en la sala de depurado, principalmente de comunicación con las bases de datos, una vez solventados estos problemas en la sala de depurado se comprobó que también hubiesen sido solventados en la interfaz del usuario.