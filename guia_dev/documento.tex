\documentclass[11pt]{article}

\usepackage[utf8]{inputenc}
\usepackage[spanish]{babel}

% \usepackage{graphicx}
% \usepackage{longtable}
% \usepackage{float}
% \usepackage{wrapfig}
% \usepackage{float}
% \usepackage{soul}
% \usepackage{amssymb}

% Colores
\usepackage{color}
\definecolor{prueba}{rgb}{.1,.1,.4}

% Resaltado de URL
\usepackage{hyperref}
\hypersetup{colorlinks=true, linkcolor=prueba,citecolor=prueba, filecolor=prueba, menucolor=prueba, urlcolor=prueba}

% Control de la separación entre párrafos
\usepackage[parfill]{parskip}

\title{Guía del desarrollador de AssessMediaWiki}
\author{José Tomás Tocino García - \texttt{josetomas.tocino@uca.es}}
\date{Marzo de 2012}

% Resaltado de código
\usepackage{minted}

% Definición de márgenes
\usepackage[top=2cm,bottom=3cm,left=2cm,right=2cm]{geometry}

% Interlineado
\usepackage{setspace}
\onehalfspacing

% Entorno propio para código de php
\newminted{php}{startinline=true}
\newminted{sql}{}

%%
%% INICIO DEL DOCUMENTO
%%
\begin{document}

\begin{titlepage}
\maketitle  
\end{titlepage}

\section{Introducción a AssessMediaWiki}

\textbf{AssessMediaWiki} es una aplicación web de código abierto capaz de
conectarse a una instalación de MediaWiki y proporcionar una serie de
herramientas de autoevaluación, heteroevaluación y evaluación entre
iguales. Todas esas evaluaciones se mantienen registradas y es posible
supervisarlas. 

AssessMediaWiki implementa dos roles de usuario distintos: supervisores y
estudiantes. Los estudiantes pueden elegir entre distintas opciones: evaluar una
revisión, comprobar sus propias aportaciones evaluadas y verificar las
evaluaciones ya enviadas. Por otro lado, los supervisores tienen un mayor número
de opciones, como modificar los parámetros de los programas o vigilar las
evaluaciones que los alumnos vayan haciendo.

Este trabajo ha sido parcialmente financiado por el Proyecto de Innovación
Docente \textbf{``La Heteroevaluación como Apoyo a la Sostenibilidad en Evaluaciones
Complejas de Trabajos Colaborativos en Wikis''} (Código PI2\_12\_029) de la
Universidad de Cádiz

\subsection{Bases tecnológicas}

AssessMediaWiki está desarrollado en el lenguaje de programación
\textbf{PHP}\footnote{\url{http://www.php.net}} y utilizando el framework
\textbf{CodeIgniter}\footnote{\url{http://codeigniter.com}}. Se trata de un
framework de desarrollo web que utiliza el patrón
\textbf{Modelo-Vista-Controlador}\footnote{\url{http://es.wikipedia.org/wiki/Modelo_vista_controlador}},
que divide las aplicaciones web en tres partes, de forma que el mantenimiento y
las actualizaciones posteriores sean más sencillas:

\begin{itemize}
\item Los \textbf{modelos} representan los datos de la aplicación. Por regla
  general un modelo corresponde a una entidad (por ejemplo, un
  \textit{usuario}), y se le suele asignar una tabla en la base de datos. Los
  modelos también albergan métodos para el trabajo con los datos, abstrayendo
  los pormenores del acceso al usuario.
\item Las \textbf{vistas} son las formas en las que se presentan los datos al
  usuario. En la mayoría de los casos, las vistas son plantillas web hechas en
  HTML y CSS, pero también es posible representar los datos utilizando vistas en
  otros formatos, como XML o JSON.
\item Los \textbf{controladores} se sitúan entre los modelos y las vistas, y
  ejecutan la lógica de la aplicación. Reciben las peticiones del navegador,
  deciden qué operaciones deben realizar y qué datos necesitan, y cargan las
  vistas apropiados con los datos necesarios. La mayor parte de la programación
  se ubica en los controladores.
\end{itemize}

\section{Instalación}

La instalación de AssessMediaWiki parte de una \textbf{instalación previa de
  MediaWiki}. Es importante que el usuario conozca los datos de acceso a la base
de datos de MediaWiki. Para más información sobre cómo instalar MediaWiki u
obtener las credenciales de la base de datos, visite la web de
MediaWiki\footnote{\url{http://www.mediawiki.org/wiki/MediaWiki}}.

El primer paso es crear una base de datos para AMW, y un usuario que tendrá
todos los privilegios sobre ella. Se reflejarán los pasos a seguir desde la
línea de comandos de un sistema GNU/Linux, aunque el procedimiento también puede
hacerse de manera visual mediante asistentes como \textit{PhpMyAdmin}. Para
acceder a la terminal de mysql utilizaremos:

\begin{sqlcode}
% mysql -u root -p
\end{sqlcode}

Una vez dentro, ejecutaremos el siguiente comando para crear la base de datos
\texttt{datos\_amw}.

\begin{sqlcode}
CREATE DATABASE datos_amw;   
\end{sqlcode}

Seguidamente, crearemos el usuario \texttt{'usuario\_amw'} y le daremos permisos sobre la base de datos.

\begin{sqlcode}
GRANT ALL ON datos_amw.* TO usuario_amw@localhost IDENTIFIED BY 'clave_amw';  
\end{sqlcode}

Hecho esto, iremos al fichero \texttt{application/config/database.php} y
cambiaremos los parámetros de acceso, de forma que coincidan con los que hemos
utilizado:

\begin{phpcode}
$db['default']['username'] = 'usuario_amw';
$db['default']['password'] = 'clave_amw';
$db['default']['database'] = 'datos_amw';  
\end{phpcode}
%$

El siguiente paso es configurar el acceso a la base de datos de MediaWiki. Para
ello, editamos el fichero \texttt{application/config/amw.php} con los datos de acceso:

\begin{phpcode}
$config["database_mw"] = "bd_mw";
$config["username_mw"] = "usuario_mw";
$config["password_mw"] = "clave_mw";
\end{phpcode}
%$

El siguiente paso es indicar qué usuarios tendrán permisos de administrador. Por
regla general, los profesores de la asignatura y el desarrollador deberían tener
acceso de administrador. Previamente es necesario conocer los ID numéricos de
cada uno de los usuarios administradores, que podremos encontrar en la tabla
\texttt{users} de MediaWiki. Seguidamente, iremos al fichero
\texttt{application/config/amw.php} e incluiremos en la clave
\textit{usuarios\_admin} los ID de los usuarios administradores:

\begin{phpcode}
// Administradores los usuarios con ID = 1 y ID = 2
$config["usuarios_admin"] = array(1, 2);  
\end{phpcode}
%$

Con esto ya podremos acceder a la aplicación, para lo que habrá que usar las
credenciales de acceso al wiki. Inicialmente AMW se encontrará en \textit{modo
  desarrollo}, de forma que al hacer login no hará falta meter la contraseña
exacta del usuario con el que queramos acceder, cualquier contraseña
valdrá. Esto nos permitirá probar los diferentes roles de usuario a la hora de
hacer pruebas.

\textbf{ES INDISPENSABLE} que al poner la aplicación a disposición de los
usuarios se desactive el modo desarrollo. Para ello, accederemos al fichero
\texttt{application/config/amw.php} y pondremos el valor a \texttt{FALSE}.

\begin{phpcode}
$config["modo_desarrollo"] = FALSE; 
\end{phpcode}
%$

\section{Componentes }

\subsection{Orígenes de datos}

Para poder trabajar, AssessMediaWiki hace uso de dos orígenes de datos. En
primer lugar, la aplicación tiene su \textbf{propia base de datos}, en la que se
guardan las evaluaciones de las aportaciones del wiki, las revisiones de las
evaluaciones y los comentarios, así como diversos parámetros de evaluación.

Por otro lado, la aplicación debe tener acceso a la \textbf{base de datos del
  wiki} sobre el que se van a hacer las evaluaciones. Además, toda la
autenticación se hace sobre la base de usuarios del wiki, por lo que evitamos
que haya que hacer dos registros.

\subsection{Modelos}

\subsubsection{Acceso}

El modelo de acceso (\texttt{application/models/acceso\_model.php}) se encarga de
establecer la conexión a la base de datos de MediaWiki. Al utilizar este modelo,
automáticamente establecerá una conexión con la base de datos MySQL de MediaWiki
que se haya configurado en la aplicación.

Otra de las tareas de este modelo es obtener la información de acceso a la
aplicación, haciendo uso de la tabla de usuarios de MediaWiki. El
\textbf{controlador de acceso} hará uso de estas capacidades para gestionar el
login.

\subsubsection{Category}

AssessMediaWiki trabaja sobre un subconjunto de los artículos presentes en el
wiki MediaWiki. En particular, la revisión se produce sobre los artículos de una
categoría en particular, que puede configurarse desde la aplicación

El modelo \textit{Category} (\texttt{application/models/category\_model.php})
lee y gestiona esta categoría, que se encuentra almacenada en la BD de AMW como
un registro de la tabla \texttt{config} cuyo campo \texttt{parameter} tiene el
valor \texttt{category}.

\subsubsection{CSV}

\subsubsection{Entregable}

A la hora de evaluar una edición en un artículo se definen una serie de
\textbf{conceptos} a evaluar. Cada uno de estos conceptos es representado como
un \textit{entregable}, con un título y una descripción.

El modelo \textit{Entregable}
(\texttt{application/models/entregable\_model.php}) se encarga de leer los
entregables establecidos en la base de datos.

\subsubsection{Evaluación}

Cuando un usuario hace una evaluación sobre una revisión de un artículo, por
cada uno de los conceptos evaluables que ha definido el profesor (representados
por el modelo \textit{Entregable}) el usuario hace una evaluación numérica y
tiene la posibilidad de dejar un comentario escrito. Para representar esta
información en la base de datos existen dos tablas. 

En primer lugar, la tabla \texttt{evaluaciones} indica quién está haciendo la
evaluación, qué revisión se está evaluando y cuál fue su autor. Seguidamente, la
tabla \texttt{evaluaciones\_entregables} asocia cada una de esas evaluaciones a
los conceptos que han sido evaluados junto a la nota numérica y el comentario
que el evaluador puede haber añadido.

El modelo \textit{Evaluaciones}
(\texttt{application/models/evaluaciones\_model.php}) se encarga de gestionar
toda esta información en la base de datos. 

\begin{itemize}
\item El método \texttt{consultar\_entregables} devuelve la información sobre
  los entregables evaluados relacionados con una evaluación en particular, cuya
  ID se debe proveer.
\item El método \texttt{listado} devuelve el conjunto de revisiones de artículos
  que ya han sido evaluados.
\item El método \texttt{evaluados}.
\end{itemize}

\subsubsection{Reply}

Cuando un usuario no está conforme con una evaluación que han hecho sobre una
aportación suya, o cuando un evaluador quiere modificar a posteriori una
evaluación ya enviada, es posible añadir una réplica, que se añade como una
evaluación más. Para representar la relación entre una evaluación y sus
evaluaciones \textit{hijas} o \textit{replies}, se utiliza el modelo
\textit{Reply} (\texttt{application/models/reply\_model.php}).

Este modelo se enlaza con la tabla \texttt{replies}, que simplemente modela una
relación 1:N entre evaluaciones \textit{padre} y sus réplicas. El modelo ofrece
varias funciones para guardar replies y obtener las replies asociadas a una
evaluación.

\subsubsection{Revisión}

\subsubsection{Usuarios}

Este modelo (\texttt{application/models/usuarios\_model.php}) establece una
conexión con la BD de MediaWiki y almacena el ID y nombre de todos los usuarios
registrados.

Además, dado un ID de un usuario se encarga de comprobar, mediante su método
\texttt{admin}, si el usuario indicado es administrador o no.

% ----------------------------------------------------------------------------------------------------

\subsection{Controladores}

Como se ha comentado previamente, los controladores dirigen la lógica de la
aplicación, haciendo uso de los datos contenidos en los \textit{modelos} y
mostrándolos mediantes las \textit{vistas}. Los controladores de los que dispone
la aplicación son los siguientes:

\subsubsection{Acceso}

El controlador de acceso (\texttt{application/controllers/acceso.php}) se
encarga de controlar el acceso de los usuarios a la aplicación. 

Su método \texttt{index} muestra y gestiona el formulario de login a la
aplicación. Los usuarios que se utilizan en el acceso a AMW son los presentes en
la tabla de usuarios de MediaWiki, y esta lectura se hace a través de los
métodos del modelo de acceso. Una vez superado el login, el usuario es
redirigido a la acción principal del controlador \texttt{Evaluar}.

Su método \texttt{salir} borra la información de login y redirige el usuario al
panel de acceso inicial.

El controlador de acceso está definido en el fichero de configuración de rutas
(\texttt{application/config/routes.php}) como el \textbf{controlador por
  defecto}, de forma que al acceder a la aplicación sin indicar ningún
parámetro, será este controlador el que se haga cargo de la petición.

\subsubsection{Alumnos}

El controlador \textit{Alumnos} (\texttt{application/controllers/alumnos.php})
muestra una lista de todos los alumnos presentes en el wiki (y, por extensión,
en AMW), y permite acceder a información sobre las acciones de estos. En
concreto, muestra enlaces a las revisiones hechas por cada alumno así como a un
documento CSV\footnote{CSV - Comma Separated Values} con información que puede
ser usada para minería de datos.

Este controlador solo permite el acceso a los usuarios que sean
administradores. El resto de usuarios es redirigido al panel principal de
evaluación.

\subsubsection{Evaluar}

\subsubsection{Feedback}

El controlador \textit{Feedback} (\texttt{application/controllers/feedback.php})
se encarga de mostrar información sobre las acciones que han realizado los
usuarios en la aplicación. En su acción por defecto (tanto \texttt{index} como
\texttt{informe}), muestra una lista de todas las evaluaciones que se han hecho
en forma de tabla, con una columna para el número de revisión evaluada y otra
columna con un enlace a la información de la revisión.

El método 

\subsubsection{Params}

\section{Mejoras}

Unificar acceso a lista de de usuarios, en algunos lados se accede directamente
al array usuarios y en otros lados se utiliza la función users del modelo usuarios

Añadir comprobaciones cuando se pidan objetos inexistentes


\end{document}