\documentclass[11pt]{article}

\usepackage[utf8]{inputenc}
\usepackage[spanish]{babel}

% \usepackage{graphicx}
% \usepackage{longtable}
% \usepackage{float}
% \usepackage{wrapfig}
% \usepackage{float}
% \usepackage{soul}
% \usepackage{amssymb}

% Colores
\usepackage{color}
\definecolor{prueba}{rgb}{.1,.1,.4}

% Resaltado de URL
\usepackage{hyperref}
\hypersetup{colorlinks=true, linkcolor=prueba,citecolor=prueba, filecolor=prueba, menucolor=prueba, urlcolor=prueba}

% Control de la separación entre párrafos
\usepackage[parfill]{parskip}

\title{Guía del desarrollador de AssessMediaWiki}
\author{José Tomás Tocino García - \texttt{josetomas.tocino@uca.es}}
\date{Marzo de 2012}

% Resaltado de código
\usepackage{minted}

% Definición de márgenes
\usepackage[top=2cm,bottom=3cm,left=2cm,right=2cm]{geometry}

% Interlineado
\usepackage{setspace}
\onehalfspacing

% Entorno propio para código de php
\newminted{php}{startinline=true}
\newminted{sql}{}

%%
%% INICIO DEL DOCUMENTO
%%
\begin{document}

\begin{titlepage}
\maketitle  
\end{titlepage}

\section{Introducción a AssessMediaWiki}

\textbf{AssessMediaWiki} es una aplicación web de código abierto capaz de
conectarse a una instalación de MediaWiki y proporcionar una serie de
herramientas de autoevaluación, heteroevaluación y evaluación entre
iguales. Todas esas evaluaciones se mantienen registradas y es posible
supervisarlas. 

AssessMediaWiki implementa dos roles de usuario distintos: supervisores y
estudiantes. Los estudiantes pueden elegir entre distintas opciones: evaluar una
revisión, comprobar sus propias aportaciones evaluadas y verificar las
evaluaciones ya enviadas. Por otro lado, los supervisores tienen un mayor número
de opciones, como modificar los parámetros de los programas o vigilar las
evaluaciones que los alumnos vayan haciendo.

Este trabajo ha sido parcialmente financiado por el Proyecto de Innovación
Docente \textbf{``La Heteroevaluación como Apoyo a la Sostenibilidad en Evaluaciones
Complejas de Trabajos Colaborativos en Wikis''} (Código PI2\_12\_029) de la
Universidad de Cádiz

\section{Vista general técnica}

\subsection{Bases tecnológicas}

AssessMediaWiki está desarrollado en el lenguaje de programación
\textbf{PHP}\footnote{\url{http://www.php.net}} y utilizando el framework
\textbf{CodeIgniter}\footnote{\url{http://codeigniter.com}}. Se trata de un
framework de desarrollo web que utiliza el patrón
\textbf{Modelo-Vista-Controlador}\footnote{\url{http://es.wikipedia.org/wiki/Modelo_vista_controlador}},
que divide las aplicaciones web en tres partes, de forma que el mantenimiento y
las actualizaciones posteriores sean más sencillas:

\begin{itemize}
\item Los \textbf{modelos} representan los datos de la aplicación. Por regla
  general un modelo corresponde a una entidad (por ejemplo, un
  \textit{usuario}), y se le suele asignar una tabla en la base de datos. Los
  modelos también albergan métodos para el trabajo con los datos, abstrayendo
  los pormenores del acceso al usuario.
\item Las \textbf{vistas} son las formas en las que se presentan los datos al
  usuario. En la mayoría de los casos, las vistas son plantillas web hechas en
  HTML y CSS, pero también es posible representar los datos utilizando vistas en
  otros formatos, como XML o JSON.
\item Los \textbf{controladores} se sitúan entre los modelos y las vistas, y
  ejecutan la lógica de la aplicación. Reciben las peticiones del navegador,
  deciden qué operaciones deben realizar y qué datos necesitan, y cargan las
  vistas apropiados con los datos necesarios. La mayor parte de la programación
  se ubica en los controladores.
\end{itemize}

\subsection{Orígenes de datos}

Para poder trabajar, AssessMediaWiki hace uso de dos orígenes de datos. En
primer lugar, la aplicación tiene su \textbf{propia base de datos}, en la que se
guardan las evaluaciones de las aportaciones del wiki, las revisiones de las
evaluaciones y los comentarios, así como diversos parámetros de evaluación.

Por otro lado, la aplicación debe tener acceso a la \textbf{base de datos del
  wiki} sobre el que se van a hacer las evaluaciones. Además, toda la
autenticación se hace sobre la base de usuarios del wiki, por lo que evitamos
que haya que hacer dos registros.

\section{Instalación}

La instalación de AssessMediaWiki parte de una \textbf{instalación previa de
  MediaWiki}. Es importante que el usuario conozca los datos de acceso a la base
de datos de MediaWiki. Para más información sobre cómo instalar MediaWiki u
obtener las credenciales de la base de datos, visite la web de
MediaWiki\footnote{\url{http://www.mediawiki.org/wiki/MediaWiki}}.

El primer paso es crear una base de datos para AMW, y un usuario que tendrá
todos los privilegios sobre ella. Se reflejarán los pasos a seguir desde la
línea de comandos de un sistema GNU/Linux, aunque el procedimiento también puede
hacerse de manera visual mediante asistentes como \textit{PhpMyAdmin}. Para
acceder a la terminal de mysql utilizaremos:

\begin{sqlcode}
% mysql -u root -p
\end{sqlcode}

Una vez dentro, ejecutaremos el siguiente comando para crear la base de datos
\texttt{datos\_amw}.

\begin{sqlcode}
CREATE DATABASE datos_amw;   
\end{sqlcode}

Seguidamente, crearemos el usuario \texttt{'usuario\_amw'} y le daremos permisos sobre la base de datos.

\begin{sqlcode}
GRANT ALL ON datos_amw.* TO usuario_amw@localhost IDENTIFIED BY 'clave_amw';  
\end{sqlcode}

Hecho esto, iremos al fichero \texttt{application/config/database.php} y
cambiaremos los parámetros de acceso, de forma que coincidan con los que hemos
utilizado:

\begin{phpcode}
$db['default']['username'] = 'usuario_amw';
$db['default']['password'] = 'clave_amw';
$db['default']['database'] = 'datos_amw';  
\end{phpcode}
%$

El siguiente paso es configurar el acceso a la base de datos de MediaWiki. Para
ello, editamos el fichero \texttt{application/config/amw.php} con los datos de acceso:

\begin{phpcode}
$config["database_mw"] = "bd_mw";
$config["username_mw"] = "usuario_mw";
$config["password_mw"] = "clave_mw";
\end{phpcode}
%$

Con esto ya podremos acceder a la aplicación, para lo que habrá que usar las
credenciales de acceso al wiki. Inicialmente AMW se encontrará en \textit{modo
  desarrollo}, de forma que al hacer login no hará falta meter la contraseña
exacta del usuario con el que queramos acceder, cualquier contraseña
valdrá. Esto nos permitirá probar los diferentes roles de usuario a la hora de
hacer pruebas.

\end{document}